% Use documentclass{article}
\documentclass{report}


% ==================================================================================================================================
% HEADER
% Documents packages, environements, page configuration

\usepackage[french]{babel}
\usepackage[T1]{fontenc}
\usepackage{ragged2e}
\usepackage{amsfonts}
\usepackage{systeme}
\usepackage{amsmath}
\usepackage{amssymb}
\usepackage{comment}
\usepackage{multicol}
\usepackage{lipsum} 
\usepackage{graphicx}
\usepackage{stmaryrd}
\usepackage{wrapfig}
\usepackage{colortbl}
\usepackage{cellspace}
\usepackage{ntheorem}
\usepackage{lmodern}
\usepackage{mathtools}
\usepackage{ragged2e}
\usepackage{tabularx}
\usepackage{titlepic}
\usepackage{fancyhdr}
\usepackage{caption}
\usepackage{xcolor} % pour les couleurs
\usepackage[linkbordercolor=white]{hyperref} % après avoir chargé xcolor
\usepackage[T1]{fontenc}
\usepackage{lmodern}
\usepackage{listings}
\usepackage{tikz}
\usepackage{mdframed}
\usepackage{xparse} % Nécessaire pour définir des environnements avec arguments optionnels
\usepackage[top=3cm, bottom=3cm, left=3.5cm, right=3.5cm]{geometry}
\usepackage{needspace} % Pour éviter le fractionnement des environnnements
% \usepackage{parskip} % Désactive l'indentation des pragraphes et rajoute un espace entre deux 
\usetikzlibrary{patterns, automata, positioning} % Pour utiliser les motifs
\usepackage{tocloft} % Permet de personnaliser le sommaire
\usepackage{minitoc} % Pour les mini tables des matières
\usepackage{forest} % Pour les arbres de dérivation



% \setcounter{secnumdepth}{-1} % Désactive le compteur des parties
\setcounter{tocdepth}{1} % Affiche uniquement les sections dans le sommaire principal
\dominitoc % Active les sommaires locaux

% Rediriger les fichiers auxiliaires de minitoc dans un dossier "aux/"
\makeatletter
\def\MTCTemp#1{aux/#1}
\makeatother

% PAGE SETTINGS


\setlength{\columnseprule}{1pt}               % ligne de séparation entre les colonnes
\def\columnseprulecolor{\color{black}}

\newcolumntype{C}{>{$\displaystyle}Sc<$}      % style d'affichage et taille 
\cellspacetoplimit=5pt                        % de la séparation entre colonnes
\cellspacebottomlimit=5pt


% MATHS SHORTHANDS

\newcommand{\C}{\mathbb{C}}
\newcommand{\R}{\mathbb{R}}
\newcommand{\Q}{\mathbb{Q}}
\newcommand{\Z}{\mathbb{Z}}
\newcommand{\N}{\mathbb{N}}
\newcommand{\U}{\mathbb{U}}
\newcommand{\K}{\mathbb{K}}
\newcommand{\M}{\mathcal{M}}
\newcommand{\B}{\mathcal{B}}

\let\oldphi\phi % Sauvegarder l'ancienne définition de \phi
\renewcommand{\phi}{\varphi} % Redéfinir \phi pour qu'il affiche \varphi


% MOD NOTATION

\theorembodyfont{\upshape}

% Définir un environnement pour encadrer les définitions avec un titre
\NewDocumentEnvironment{definition}{O{}}
{
  \begin{mdframed}[linewidth=0pt,linecolor=gray,backgroundcolor=gray!10,roundcorner=5pt]
  \textbf{Définition}%
  \IfNoValueTF{#1}{}{~(\textbf{#1})} % Affiche le titre entre parenthèses et en gras s'il est fourni
  . % Point à la fin
}
{
  \end{mdframed}
}

% Définir un environnement pour encadrer les théorèmes avec un titre
\NewDocumentEnvironment{theorem}{O{}}
{
  \samepage
  \begin{mdframed}[linewidth=1pt,linecolor=darkgray,backgroundcolor=darkgray!10,roundcorner=5pt]
  \textbf{Théorème}%
  \IfNoValueTF{#1}{}{~(\textbf{#1})} % Affiche le titre entre parenthèses et en gras s'il est fourni
  . % Point à la fin
}
{
  \end{mdframed}
}

% Définir un environnement pour encadrer les corollaires
\NewDocumentEnvironment{corollary}{O{}}
{
  \begin{mdframed}[linewidth=1pt,linecolor=gray,backgroundcolor=gray!10,roundcorner=5pt]
  \textbf{Corollaire}%
  \IfNoValueTF{#1}{}{~(\textbf{#1})} % Affiche le titre entre parenthèses et en gras s'il est fourni
  . % Point à la fin
}
{
  \end{mdframed}
}


% Définir un environnement pour encadrer les critères
\NewDocumentEnvironment{criteria}{O{}}
  {
    \begin{mdframed}[linewidth=1pt,
                      linecolor=gray,
                      backgroundcolor=gray!10,
                      roundcorner=5pt
                      ]
    \textbf{Critère}%
    \IfNoValueTF{#1}{}{~(\textbf{#1})} % Affiche le titre entre parenthèses et en gras s'il est fourni
    . % Point à la fin
  }
  {
    \end{mdframed}
  }

% Définir un environnement pour encadrer les prorpiétés avec un titre
\NewDocumentEnvironment{prop}{O{}}
{
  \begin{mdframed}[linewidth=0pt,linecolor=gray,backgroundcolor=gray!10,roundcorner=5pt]
  \textbf{Propriété}%
  \IfNoValueTF{#1}{}{~(\textbf{#1})} % Affiche le titre entre parenthèses et en gras s'il est fourni
  . % Point à la fin
}
{
  \end{mdframed}
}
				

\theoremstyle{plain}
\newtheorem*{remark}{Remarque}
\newtheorem*{proposition}{Proposition}
\newtheorem*{lemma}{Lemme}
%\newtheorem*{prop}{Propriété}
\newtheorem*{proof}{Démonstration}
\newtheorem*{example}{Exemple}

% Créer un environnement qui empêche les coupures de page
\newenvironment{nobreakproposition}
{
  \par\begingroup\samepage % Empêcher les coupures de page
  \begin{proposition}
}
{
  \end{proposition}
  \par\endgroup
}



% ==================================================================================================================================
% Setting Title 

% Titre du document
\title{Théorie des Langages}
\author{Axel PIGEON}
\date{\today}

\setlength{\parindent}{0pt}
\renewcommand{\labelitemi}{\textbullet} % Utiliser des points noirs (•)

\begin{document}


% ==================================================================================================================================
% CONTENT


% Page de titre 
\maketitle

% Table des matières 
\tableofcontents
\newpage

\setlength{\parindent}{0pt}

% Inclusion des chapitres 
\chapter{Rappels sur les ensembles}

\minitoc % Ajoute le sommaire local ici

\justify

\setlength{\parindent}{0pt}

% ==================================================================================================================================
% Rappels

Rappels sur les ensembles
\chapter{Mots et Langages}

\minitoc % Ajoute le sommaire local ici

\justify

\setlength{\parindent}{0pt}
\renewcommand{\labelitemi}{\textbullet} % Utiliser des points noirs (•)

% ==================================================================================================================================
% Introduction 

Si on connaît plusieurs langages de programmation, on remarque que chaque langage, ou plutôt chaque paradigme de 
langage est spécialisé pour la résolution d'une carégorie dde problèmes. 
On pourrait se demander s'il est possible de créer un langage permettant de résoudre tous les problèmes. 
Pour cela, il nous faudrait d'abord être capable de définir formellement un langage, des mots, etc... 

% ==================================================================================================================================
% Alphabets et Mots 

\section{Alphabets et Mots}

\subsection{Premières Définitions}

Commençons tout d'abord par redéfinir correctement la notion d'alphabet et de mot. 

\begin{definition}[Alphabet]
    Un alphabet est simplement un ensemble fini, noté $\Sigma$.
    On nomme "lettre" ou "symbole" les éléments d'un alphabet.  
\end{definition}

\begin{example}
    Quelques exemples d'alphabets :
    \begin{itemize}
        \item $\Sigma = \{a,b\}$ 
        \item $\Sigma = \{a,b,\dots,\text{\%},\text{\$}\}$
    \end{itemize}
\end{example}

\begin{definition}[Mot]
    Un mot est une suite finie de lettres d'un alphabet. 
\end{definition}

\begin{proposition}
    Le mot vide est noté $\varepsilon$. On note l'ensemble des mots d'un alphabet $\Sigma^*$. 
\end{proposition}

\begin{definition}[Longueur d'un mot]
    On note $|w|$ la longueur d'un mot $w \in \Sigma^*$ qui correspond au nombre de lettres, avec répétition du mot $w$. 
\end{definition}

\begin{definition}[Egalité de mots]
    On dit que deux mots $u,v \in \Sigma^*$ son égaux ssi :
    \begin{itemize}
        \item $|v| = |u|$ 
        \item $\forall i \in \llbracket 1, |v| \rrbracket, \quad u[i] = v[i]$ où $u[i]$ est la ième lettre du mot $u$. 
    \end{itemize}
    \emph{Deux mot sont égaux ssi ils sont de même longueur et sont composés des mêmes lettres dans le même ordre.}
\end{definition}

\subsection{Opérations sur les mots}

Sur les mots, on ne définit qu'une seule opération, la \textbf{concaténation}. 

\begin{definition}[Concaténation]
    Soient $u,v \in \Sigma^*$ deux mots définis sur un même alphabet. On appelle la concaténation l'application :
    \[
        \begin{cases}
            \Sigma^* \times \Sigma^* \longrightarrow \Sigma^* \\ 
            (u,v) \longmapsto w = u.v
        \end{cases}
    \]
    Elle est définie telle que $ \forall u,v \in \Sigma^*$ de longueur $n,p \in \N$, on ait :
    \begin{itemize}
        \item $ |u.v| = |u| + |v| = n + p $
        \item $ \forall i \in \llbracket 1, n \rrbracket u.v[i] = u[i] $ et $ \forall i \in \llbracket 1,p \rrbracket, u.v[n+i] = v[i] $  
    \end{itemize}
    On parlera identiquement de concaténation ou de produit. 
\end{definition}

\begin{remark}
    Cette définition reprend bien la caractérisation de deux mots égaux. 
\end{remark}

\begin{proposition}[Propriétés de la concaténation]
    La concaténation est une application :
    \begin{itemize}
        \item \textbf{Associative : } $\forall u,v,w \in \Sigma^*, \quad w.(u.v) = (w.u).v$ 
        \item \textbf{Pas commutative} pour un alphabet de plus d'une lettre. 
        \item admet pour \textbf{élément neutre} le mot vide $\varepsilon$. 
    \end{itemize}
\end{proposition}

\subsection{Puissance d'un mot}

Une fois la concaténation définie pour un mot, on peut alors parler de puissance de mot. 
Définissons celle-ci par récurrence. 

\begin{definition}[Puissance d'un mot]
    Soit $\Sigma$ un alphabet et $ u \in \Sigma^*$, on a :
    \begin{itemize}
        \item $u^0 = \varepsilon $
        \item $u^1 = u$ 
        \item $ \forall n \in \N, u^{n+1} = u^n.u$ 
    \end{itemize}
\end{definition}

\begin{example}
    $(ba)^3 = bababa $
\end{example}

\begin{proposition}
    Soient $u \in \Sigma^*$ on peut appliquer les règles "connues" des puissances d'où : $$ \forall n,p \in \N, \quad  u^{n+p} = u^n . u^p = u^p . u^n $$ 
\end{proposition}

On remarque que l'on peut effectuer des simplifications sur les égalités de mots. 

\begin{prop}[Simplifications]
    L'ensemble $\Sigma^*$ est simplifiable à gauche et à droite. 
    \begin{itemize}
        \item $\forall u,v,w \in \Sigma^*, \quad u.w = v.w \Longrightarrow u = v $
        \item $\forall u,v,w \in \Sigma^*, \quad w.u = w.v \Longrightarrow u = v $
    \end{itemize}
\end{prop}

Ici, pas besoin d'inverse, la démonstration repose sur la définition de l'égalité entre deux mots. 

% ==================================================================================================================================
% Ordre

\newpage 

\section{Relations d'Ordre}

Dans l'alphabet dit "classique" on possède un ordre lexicographique des mots permettant de les classer 
en fonction de leurs lettres et de la position de leurs lettres. 
Ici, nous allons définir deux types de relations d'ordre sur les mots. 

Commençons par rappeler la définition de relation d'ordre. 

\begin{definition}[Relation d'Ordre]
    Soit $\lhd $ une relation sur un ensemble $E$. On dit que $\lhd$ est une \textbf{relation d'ordre} ssi 
    pour tout $x,y,z \in E$, $\lhd$ est :
    \begin{itemize}
        \item \textbf{Réflexive : } $ x \lhd x $ 
        \item \textbf{Anti-Symétrique : } $ x \lhd y \text{ et } y \lhd x \Longrightarrow x = y $ 
        \item \textbf{Transitive : } $x \lhd z \text{ et } z \lhd y \Longrightarrow x \lhd y $ 
    \end{itemize}
\end{definition}

\subsection{Ordre Préfixe}

Naturellement, on munit$ \Sigma^*$ d'un ordre préfixe permettant de classer les mots en fonction de leur préfixe. 
Cette relation peut être vue comme un forme d'inclusion de mots. 

\begin{definition}[Ordre Préfixe]
    Soient $u,w \in \Sigma^*$, on définit la relation d'ordre préfixe $\sqsubseteq$ telle que :
        \[ \boxed{u \sqsubseteq w \iff \exists v \in \Sigma^*, w = u.v} \] 
    Autrement dit, $u$ est un préfixe de $w$ ssi il existe un mot $v$ tel que $w$ soit composé de 
    la concaténation de $u$ et $v$. 
\end{definition}

\begin{remark}
    L'ordre préfixe ne nécessite pas de relation d'ordre directement sur l'alphabet $\Sigma$. 
\end{remark}

\begin{prop}[Ordre Préfixe et égalité]
    Soient $u,v \in \Sigma^*$ on a : 
        \[ u \sqsubseteq v \text{ et } v \sqsubseteq u \Longrightarrow u = v \] 
\end{prop}

\begin{quote}
    \begin{footnotesize}
        \begin{proof}
            Soient $u,v \in \Sigma^*$ tels que $ \exists x,y \in \Sigma^*$ tels que 
                \[ u = v.x \quad \text{ et } \quad v = u.y \] 
            On a alors que :
                \[
                    \begin{cases}
                        v = v.y.x \\ 
                        yx = \varepsilon 
                    \end{cases}
                    \; \Longrightarrow 
                    \begin{cases}
                        y = \varepsilon \\ 
                        x = \varepsilon 
                    \end{cases}
                    \; \Longrightarrow u = v 
                \]
        \end{proof}
    \end{footnotesize}
\end{quote}

\begin{remark}
    \textbf{Attention : } l'ordre préfixe est une relation d'ordre partielle. 
    Autrement dit, tous les éléments d'un même alphabet de sont pas comparables. 
\end{remark}

\subsection{Ordre Lexicographique}

\begin{definition}[Ordre Lexicographique]
    Soit $\Sigma$ un alphabet que l'on muni d'une relation d'ordre $\leqslant$. 
    L'ordre lexicographique $\leqslant$ est une relation d'ordre totale sur $\Sigma^*$. 
\end{definition}

\begin{remark}
    Ici, nous avons bien besoin de définir au préalable un ordre sur notre alphabet $\Sigma$. 
\end{remark}

\begin{prop}[Compatibilité]
    L'ordre lexicographique est compatible avec l'ordre préfixe. Plus formellement, 
        \[ \forall u,v \in \Sigma^*, \quad u \sqsubseteq v \Longrightarrow u \leqslant v \] 
\end{prop}



% ==================================================================================================================================
% Langage

\section{Langage}

Maintenant que nous sommes au clair sur la définition de lettre et de mot, on peut enfin définir l'objet 
principal de ce chapitre, les langages. 

\subsection{Definition}

\begin{definition}[Langage]
    Soit $\Sigma$ un alphabet, on appelle langage sur $\Sigma$ toute partie 
    de $\Sigma^*$. 
\end{definition}

\begin{remark}
    L'ensemble de tous les langages d'un alphabet $\Sigma$ est donc $\mathcal{P}(\Sigma^*)$, 
    l'ensemble de toutes les parties de $\Sigma^*$. 
\end{remark}

\begin{definition}[Complémentaire]
    Soit $L$ un langage sur $\Sigma$. On définit le complémentaire de $L$ dans $\Sigma^*$ le langage :
        \[ \overline{L} = \{w, \; w \not \in L\} \] 
\end{definition}

\subsection{Opérations}

De même que pour les alphabets et les mots, on peut définir des opérations sur les langages. 

\begin{definition}[Opérations sur les langages]
    Soit $\Sigma$ un alphabet et $L_1, L_2 \subseteq \Sigma^*$ deux langages de $\Sigma$. 
    On définit 4 principales opérations sur des langages :
    \begin{itemize}
        \item \textbf{Somme : } notée $+$, la somme de deux langages d'apparente à l'union des ensembles. 
            \[ \boxed{ L_1 + L_2 = \{w, \; w \in L_1 \text{ ou } w \in L_2\} }\] 
        C'est une opération :
            \begin{itemize}
                \item Commutative 
                \item Associative 
                \item dont $\emptyset$ est le neutre. 
            \end{itemize}
        \item \textbf{Intersection : } de même que pour les ensembles :
            \[ \boxed{ L_1 \cap L_2 = \{w, \; w \in L_1 \text{ et } w \in L_2\} }\] 
            C'est une opération : 
            \begin{itemize}
                \item Commutative 
                \item Associative 
                \item dont $\Sigma^*$ est le neutre
            \end{itemize}
        \item \textbf{Différence : } comme les ensembles, on définit la différence de langages :
            \[ \boxed{ L_1 / L_2 = \{w, \; w \in L_1 \text{ et } w \not \in L_2\} = L_1 \cap \overline{L_2} }\] 
        \item \textbf{Produit de concaténation : } de même que pour les mots, on peut généraliser le produit de concaténation 
        aux langages :
            \[ \boxed{L_1 . L_2 = \{ u.v, \; u \in L_1, v \in L_2\} }\] 
        C'est une opération :
        \begin{itemize}
            \item Associative 
            \item Distributive par rapport à l'union 
            \item D'élément neutre $\{\varepsilon\}$. 
        \end{itemize}
    \end{itemize}
\end{definition}

\begin{definition}[Puissance de langage]
    Soit L un langage, on définit \textbf{par récurrence} la puissance de L par :
    \begin{itemize}
        \item $L^0 = \{\varepsilon\}$ 
        \item $L^1 = L$
        \item $\forall n \in \N^*, \; L^n = L^{n-1}.L $
    \end{itemize}
\end{definition}

Une fois définies des opérations "simples" sur les langages, on peut en définir des plus complexes, permettant de "générer" un langage infini
à partir d'un langage fini ou infini. 

\begin{definition}[Langage plus et étoile]
    Soit L un langage sur un alphabet $\Sigma$. On définit le langage plus de L comme le langage :
        \[ L^+ = L^1 + L^2 + \dots \] 
    De même le langage étoile de $L$ est défini par :
        \[ L^* = \{\varepsilon\} + L^1 + L^2 + \dots \]
\end{definition}

\subsection{Propriétés}

Voyons quelques propriétés des langages... 

\begin{proposition}
    Soient $L_1$ et $L_2$ deux langages sur un alphabet $\Sigma$, on a les propriétés suivantes :
    \begin{itemize}
        \item $ \forall p \in \N,$ on a : 
            \[ \quad (L_1)^p . (L_2)^p \subseteq (L_1)^* . (L_2)^* \] 
        \item L'opération étoile est idempotente : 
            \[ (L^*)^* = L^* \] 
        \item $L^* = \{\varepsilon\} + L^+$ 
        \item $\varepsilon \in L \iff L^+ = L^* $ 
    \end{itemize}
\end{proposition}

\subsection{Expressions Régulières}

Lorsque l'on manipule des langages infinis, il serait appréciable d'avoir une expression pratique pour un langage 
permettant de directement voir la forme des mots qu'il contient. On définit ainsi les expressions régulières. 

\begin{definition}[Expression Régulière]
    On définit récurisevement une expression régulière sur un alphabet $\Sigma$ :
    \begin{itemize}
        \item $\varepsilon$ est une expression régulière. 
        \item $ \forall w \in \Sigma$ est une expression régulière. 
        \item Si E est un expression régulière alors $(E)$ l'est aussi. 
        \item Si $E_1$ et $E_2$ sont des expressions régulières, alors $E_1 + E_2$ l'est aussi. 
        \item Si $E_1$ et $E_2$ sont des expressions régulières, alors $E_1.E_2$ l'est aussi.
        \item Si $E$ est une expression régulière, alors $E^*$ l'est aussi. 
    \end{itemize}
\end{definition}

\begin{example}
    Voyons quelques exemples d'expressions régulières sur un alphabet $\Sigma = \{a,b\}$ :
    \[ a^*b, \quad (a+b)^*, \quad (a+b)^*ba(a+b)^* \] 
\end{example}

\begin{definition}[Langage Régulier]
    On dit qu'un langage est régulier si il peut s'écrire sous la forme d'une expression régulière. 
\end{definition}

Il sera donc préférable de travailler avec des langages réguliers. 

% ==================================================================================================================================
% Langage Décidable 

\section{Langage Décidable}

L'objectif de ce cours est bien entendu de comprendre comment fonctionne un compilateur, pour pouvoir en créer un par nous même. 
Pour rappel, on doit d'abord bien comprendre les notions de langage et de mot pour pouvoir ensuite déterminer si un ensemble de 
mots est syntaxiquement corrects lors de la compilation. 

Lors de la compilation, il faut d'abord commencer par savoir si un mot traité appartient au langage défini ou pas. 
Pour des langages finis, l'opération n'est pas compliquée, pour chaque mot il suffit de vérifier si il appartient à un ensemble fini. 
Pour des langages infinis, l'opération semble plus complexe, il va falloir trouver une manière systématique et efficace de 
définir si un mot appartient au langage ou pas. 

On appelle ce genre de problème un problème de \textbf{décision}. 

\begin{definition}[Langage Décidable]
    Un langage L est dit décidable si il existe un algorithme permettant de dire si un mot $w$ appartient 
    ou pas au langage L. 
\end{definition}

\begin{theorem}[Nombre de Langages Décidables]
    Il existe un nombre fini de langages décidables. 
\end{theorem}

Autrement dit, il existe un nombre infini de langages non décidables...


\chapter{Automate Fini Déterministe}

\minitoc % Ajoute le sommaire local ici

\setlength{\parindent}{0pt}
\renewcommand{\labelitemi}{\textbullet} % Utiliser des points noirs (•)


% ==================================================================================================================================
% Introduction 

Comme abordé dans le chapitre précédent, on cherche une méthode pratique et efficace pour déterminer si un mot appartient 
à un langage ou pas. On veut donc un modèle qui soit d'une part très pratique mathématiquement pour nous permettre de 
démontrer des choses dessus mais aussi facilemet implémentable algorithmiquement. 

Alerte Spoiler : de solides connaissances en théorie des graphes seront plus qu'utiles...


% ==================================================================================================================================
% Définition et représentation

\section{Définition et représentation}

\begin{definition}[Automate fini déterministe]
    Un Automate Fini Déterministe est un quintuplet :
        \[ \boxed{ \mathcal{A} = (\Sigma, Q, T, q_0, A) } \] 
    où :
    \begin{itemize}
        \item $\Sigma$ est un alphabet 
        \item $Q$ est un ensemble fini d'états (souvent une partie finie de $\N$)
        \item $T : Q \times \Sigma \longrightarrow Q$ est une application qui, à un état et une lettre associe un autre état. 
        \item $q_0$ un état initial
        \item $A \subseteq Q$ les états acceptants
    \end{itemize}
\end{definition}

On représentera ainsi un automate fini déterministe de plusieurs façons en fonction de son utilisation :
\begin{itemize}
    \item \textbf{Mathématique : } $\mathcal{A} = (\Sigma, Q, T, q_0, A) $
    \item \textbf{Table de Transition : } Elle va permettre de trouver rapidement les différents types d'états. 
    \item \textbf{Sagitale : } Sous forme de graphe 
    \item \textbf{En Python : } Nous représenterons les automates finis déterministes sous la forme de quintuplet aussi. 
\end{itemize}
Regardons en détail ces différentes représentations au travers d'un exemple. 

\newpage 


\begin{example}
    Soit $\mathcal{A} = (\Sigma, Q, T, q_0, A)$ un automate fini déterministe. 

    \textbf{Mathématiquement} nous avons :
    \begin{itemize}
        \item $Q = \{1,2,3\}$ 
        \item $\Sigma = \{a,b\}$ 
        \item $q_0 = 1$ 
        \item $A = \{3\}$ 
    \end{itemize}
    
    La représentation \textbf{sagitale}  de notre automate sera :
        
        \begin{center}
            \begin{figure}[h]
                \centering
                \begin{tikzpicture}[shorten >=1pt, node distance=2cm, on grid, auto]
    
                    % Définitions des styles d'états
                    \node[state, initial] (1) {1};
                    \node[state, right=of 1] (2) {2};
                    \node[state, accepting, right=of 2] (3) {3};
                
                    % Définition des transitions
                    \path[->]
                    (1) edge[above] node{a} (2)
                    (2) edge[above] node{b} (3)
                    (2) edge[loop above] node{a} ()
                    (1) edge[loop above] node{b} ()
                    (3) edge[loop above] node{a} ()
                    (3) edge[loop below] node{b} ();
                
                \end{tikzpicture}
            \end{figure}
        \end{center}

    On représente de façon doublement cerclée les états acceptants. Les états sont les sommets du graphe. 
    Les arcs valués sont les antécédants/images de la fonction $T$. 

    Enfin, la \textbf{table de transition} de l'automate est représentée par le tableau suivant :
    \begin{multicols}{2}
        \begin{center}
            \begin{tabular}{c|c|c}
                Q/$\Sigma$ & a & b \\ \hline 
                1 & 2 & 3 \\ \hline 
                2 & 2 & 3 \\ \hline 
                \textbf{3} & 3 & 3 
            \end{tabular}
        \end{center}

        La première ligne présente les lettres de l'alphabet et la première colonne les différents états. 
        Pour chaque état, le tableau donne l'état obtenu en fonction de la lettre suivante lue. 
    \end{multicols}
\end{example}

% ==================================================================================================================================
% Mots et Langage Automatique

\section{Mot et Langage Automatique}

\subsection{Lecture d'un mot}

\begin{definition}[Lecture d'un mot]
    Soit $\Sigma$ un dictionnaire, $l \in \Sigma$ et $\mathcal{A}$ un automate. 
    On lit la lettre $l$ en dérivant d'un état $q \in Q$ vers un état $q' \in Q$ et si $T(q,l) = q'$. 
    On notera la lecture d'un mot de longueur $p$ par la lecture successive de ses lettres :
        \[ q_1 \;  \overset{l_1}{\longrightarrow} \; q_2 \; \dots \; q_{p-1} \; \overset{l_p}{\longrightarrow} \; q_p \] 
\end{definition}

Concrètement, pour la lecture d'un mot, on va partir de l'état initial et en fonction des valeurs de l'état courant et de 
la lettre lue, on va "bouger" d'un état (sommet) à un autre.

\begin{definition}[Mot refusé]
    Un mot $w \in \Sigma^*$ est dit refusé par un automate $\mathcal{A}$ si sa lecture à partir de l'état initial 
    se termine sur un état refusant ou ne se termine pas. Dans le cas contraire, $w$ est dit accepté. 
\end{definition}

\begin{definition}[Langage d'un Automate]
    Le langage d'un automate $\mathcal{A}$ est l'ensemble des mots acceptés par l'automate. 
    On le note $L(\mathcal{A})$. 
\end{definition}

On parle de \textbf{langage automatique} si il est reconnaissable par un automate. 
Deux automates sont dits \textbf{équivalents} si ils reconnaissent le même langage. 

\subsection{Automate Complet}

\begin{definition}[Automate Complet]
    Un automate $\mathcal{A}$ est dit complet si 
        \[ \forall i \in Q, \forall l \in \Sigma, \quad T(i,l) \text{ est défini} \] 
    Autrement dit, un automate est dit complet si pour toute lettre et pour tout état fixés, il est possible de 
    changer d'état dans l'automate.  
\end{definition} 

\begin{definition}[Puit]
    Un état $q \in Q$ est un puit ssi 
        \[ \forall l \in \Sigma, \quad T(q,l) = q \]
    Un puit est un état duquel on ne peut sortir. 
\end{definition}

On définit aussi la notion de piège comme un puit refusant (i.e un puit dont l'état est refusant). 

\subsection{Complémentaire}

Puisque la notion de complémentaire existe pour les langages et que les automates semblent très étroitement liés 
aux langages, on peut se demander si un automate peut admettre un complémentaire... 

Soit $\mathcal{A}$ un automate associé à un langage L. On cherche $\mathcal{A}' = \overline{\mathcal{A}}$. 
    \[ w \in \overline{\mathcal{A}} \iff w \not \in L \iff w \not \in L(\mathcal{A}) \] 
Il semble falloir que $\mathcal{A}$ soit complet. Si c'est le cas, on pourrait inverser $\mathcal{A}$ en inversant 
les sommets accpetants/refusants. 

\begin{proposition}
    Tout automate fini peut être complété par des puits refusants en un automate complet. 
\end{proposition}

\begin{theorem}[Complémentarité]
    L'ensemble des automates est stable par complémentarité. 
\end{theorem}


% TODO : Ajouter section émonder un automate 
\chapter{Automate Fini Non Déterministe}

\minitoc % Ajoute le sommaire local ici

\setlength{\parindent}{0pt}
\renewcommand{\labelitemi}{\textbullet} % Utiliser des points noirs (•)


% ==================================================================================================================================
% Introduction 

Dans le chapitre précédent nous avons vu un modèle très efficace pour vérifier l'appartenance d'un mot à un langage. 
En plus d'être facilement représentable en mémoire (i.e Python), il est facile à utiliser à la main et 
hérite de toute la théorie des graphes vue précédement ce qui en fait un très beau modèle mathématiquement parlant. 

Malgré tout cela, nous ne savons pas comment, à partir de plusieurs langages simples, constituer un automate 
reconnaissant la somme de ces langages. Nous n'avons pas défini de somme/union d'automate et celles-ci semblent 
assez difficiles vu la rigidité de notre modèle. 

Nous allons donc construire un modèle d'automate, appelé non déterministe, nous permettant de faire ces opérations 
d'union (que nous appelerons juxtaposition). Elles nous permettrons de construire des automates complexes à partir de 
somme de langages. 

% ==================================================================================================================================
% Définitions et propriétés

\section{Généralités}

\subsection{Définitions}

\begin{definition}[AFN]
    Un automate fini non déterministe $ \mathcal{A}$ est un quintuplet : 
        \[ \mathcal{A} := (Q,\Sigma, T,I,A) \] 
    tel que :
    \begin{itemize}
        \item $Q$ est l'ensemble des états de l'automate 
        \item $\Sigma$ est un alphabet 
        \item $T : Q \times \Sigma \longrightarrow \mathcal{P}(Q)$ est une application 
        \item $I \subseteq Q$ est l'ensemble des états initiaux 
        \item $A \subseteq Q$ est l'ensemble des états acceptants. 
    \end{itemize}
\end{definition}

\begin{remark}
    Plusieurs remarques concernant ce nouveau modèle. Premièrement, on remarque que l'on peut maintenant définir 
    des transitions multiples entre les états de l'automate. Deuxièmement, il existe plusieurs états initiaux. 
\end{remark}

\subsection{Lecture d'un mot}

\begin{definition}[Arbre de lecture]
    Soient $w \in \Sigma^*$, $L$ un langage sur $\Sigma$ et $ \mathcal{A}$ un automate reconnaissant $L$.
    L'arbre de lecture de $w$ par $ \mathcal{A}$ est l'arbre résultat du parcours de $ \mathcal{A}$ en fonction 
    des lettres de $w$. 
    
    Autrement dit dans l'arbre de lecture G de $w$, un noeud $a$ est le fils d'un 
    noeud $b$ si il existe une lettre $l$ de $w$ telle que $T(b,l) = a$. Une feuille de cet arbre est un état acceptant 
    ou refusant de l'automate. 
\end{definition}

\begin{definition}[Lecture acceptante]
    Soient $w \in \Sigma^*$, $L$ un langage sur $\Sigma$ et $ \mathcal{A}$ un automate reconnaissant $L$. 
    Une lecture de $w$ par $ \mathcal{A}$ est dite acceptante si \underline{il existe} un chemin 
    d'un état initial vers un état acceptant dans l'arbre de lecture de $w$ par $ \mathcal{A}$. 
\end{definition}

\begin{proposition}
    On peut dire plusieurs choses de la lecture d'un mot $w \in \Sigma^*$ par un automate $ \mathcal{A}$ :
    \begin{itemize}
        \item Si l'automate possède plusieurs états initiaux, la lecture produit un arbre de lecture pour chaque 
        état initial, nous auront donc une \underline{forêt d'arbres de lecture}. 
        \item Une lecture sera donc acceptante ssi il existe un arbre de la forêt dont au moins une des feuilles
        est un état acceptant. 
    \end{itemize}
\end{proposition}

\begin{example}
    Voyons tout cela sur un exemple. Soit $ \mathcal{A} := (Q,\Sigma, T,I,A)$ un automate fini non déterministe $ \mathcal{A}$
    sur l'alphabet $\Sigma : \{a,b\}$. Représentons notre automate sous forme de graphe orienté valué et sa table de transition :
    \begin{center}
        \begin{minipage}{0.45\textwidth} % Colonne pour le dessin
            \begin{tikzpicture}[shorten >=1pt, node distance=3cm, on grid, auto]
    
                % Définitions des styles d'états
                \node[state, initial] (1) {1};
                \node[state, accepting, right=of 1] (2) {2};
                \node[state, below=of 1] (3) {3};
                \node[state, initial, right=of 3] (4) {4};
            
                % Définition des transitions
                \path[->]
                (1) edge[above] node{$a$} (2)
                (1) edge[left] node{$a$} (3)
                (2) edge[right] node{$a$} (4)
                (3) edge[below] node{$b$} (4)
                (4) edge[sloped, below left] node{$b$} (1)
                (4) edge[loop below] node{$b$} (4);
            
            \end{tikzpicture}
        \end{minipage}%
        \hfill 
        \begin{minipage}{0.45\textwidth} % Colonne pour le tableau
            \begin{tabular}{c|c|c}
                T & a & b \\ \hline 
                $1$ & $\{2,3\}$ & X \\ \hline 
                \textcircled{2} & $4$ & X \\ \hline 
                $3$ & X & $4$ \\ \hline 
                $4$ & X & $\{1,4\}$ 
            \end{tabular}
        \end{minipage}
    \end{center}
    C'est un automate non déterministe puisqu'il contient deux transitions mutliples et deux états initiaux. 

    Posons $w := abba$, déterminons si ce mot appartient au langage $L( \mathcal{A})$. Nous allons construire 
    un seul arbre permettant d'avoir une condition suffisante de validation du mot. 

    \begin{center}
        \begin{minipage}{0.45\textwidth}
            \begin{tikzpicture}[scale=0.8, transform shape, shorten >=1pt, node distance=3cm, on grid, auto]
                
                % Racine
                \node[state, initial] (1) {1};
    
                % 1er niveau
                \node[state] (2) [below right=2cm and 2cm of 1] {2};
                \node[state] (3) [below left=2cm and 2cm of 1] {3};
    
                % 2nd niveau
                \node[state] (x1) [below=2cm of 2] {X};
                \node[state] (4) [below=2cm of 3] {4};
                
                % 3eme niveau 
                \node[state] (42) [below left=2cm and 2cm of 4] {4};
                \node[state] (12) [below right=2cm and 2cm of 4] {1};
    
                % 4 eme niveau 
                \node[state] (22) [below right=2cm and 2cm of 12] {2};
                \node[state] (32) [below left=2cm and 2cm of 12] {3};
                \node[state] (x2) [below=2cm of 42] {X}; 
    
                % Branches
                \path[->]
                (1) edge[left] node{a} (2)
                (1) edge[left] node{a} (3)
                (2) edge[left] node{b} (x1)
                (3) edge[left] node{b} (4)
                (4) edge[left] node{b} (12)
                (4) edge[left] node{b} (42)
                (12) edge[left] node{a} (22)
                (12) edge[left] node{a} (32)
                (42) edge[left] node{a} (x2);
    
            \end{tikzpicture}
        \end{minipage}
        \hfill 
        \begin{minipage}{0.45\textwidth}
            L'arbre de lecture du mot $abba$ contient un état acceptant comme feuille. 

            \vspace{1cm}

            Autrement dit, il existe un chemin menant d'un état initial à un état acceptant dans la forêt de lecture 
            de $abba$. Donc $abba \in L( \mathcal{A})$. 
        \end{minipage}        
    \end{center}
\end{example}

% ==================================================================================================================================
% Juxtaposition et construction d'un AFD

\section{Juxtaposition et construction d'un AFD}

\subsection{Juxtaposition}

Rappelons la problématique principale du chapitre. On cherche un modèle dérivant des AFD nous permettant de définir 
des opérations dessus et qui puisse être convertit algorithmiquement vers un AFD pour construire des automates 
d'un langage complexe à partir de langages plus simple. 

Autrement dit, on veut pouvoir appliquer l'opération de somme de langages sur les automates fini déterministes. 

\begin{definition}[Juxtaposition d'AFN]
    Soient $L_1$ et $L_2$ deux langages reconnus par deux automates $ \mathcal{A}_1$ et $ \mathcal{A}_2$. 
    Le langage $L_1 + L_2$ est reconnu par la \textbf{juxtaposition disjointe} de $ \mathcal{A}_1$ et $ \mathcal{A}_2$. 
\end{definition}

\begin{theorem}[Langages automatiques et stabilité]
    L'ensemble des langages automatiques est stable par somme. 
\end{theorem}

On peut maintenant, à partir de deux langages automatiques, définir le nouveau langage résultant de 
la somme des deux qui sera lui aussi automatique. Il suffit de faire la juxtaposition disjointe des deux automates
de départ. 

\subsection{Déterminisation}

\begin{theorem}[Existence et équivalence]
    Pour tout automate fini non déterministe, il existe un automate déterministe équivalent. 
\end{theorem}

Ce théorème est peut être un peu obscur mais permet de dire qu'il est toujours possible de passer d'un 
automate fini non déterministe (obtenu par exemple par juxtaposition) à un automate fini déterministe
qui reconnaisse le même langage. En tout cas, il nous dit qu'il en existe un...

\vspace{0.3cm}

L'intérêt de vouloir repasser chez les automates fini déterministes vient du fait que la lecture d'un mot par 
un AFN est de complexité exponentielle alors que la lecture d'un mot par un AFD est polynômiale... 
Lors de la vérification syntaxique de très long mots pour des langages très complexes, cela fait une différence 
cruciale pour la compilation. 

Ce processus est appelé \textbf{déterminisation} d'un AFN. 

\begin{proposition}
    Soit $ \mathcal{A} = (Q, \Sigma, T, I, A)$ un AFN. On cherche à construire un AFD $ \mathcal{A}'$ équivalent à $ \mathcal{A}$. 
    L'idée est de raisonner sur l'application $T : Q \times \Sigma \longrightarrow \Q$. Dans un AFN, cette application 
    n'est pas injective, on va donc poser une nouvelle application dans l'espace quotient de $Q \times \Sigma$ par le noyau de $T$. 
    Nous obtiendrons donc une application injective et donc un AFD. 
\end{proposition}

\newpage 

\begin{example}[Déterminisation]
    Soit $ \mathcal{A}$ l'automate défini sur l'alphabet $\Sigma = \{ a,b \}$ non déterministe et sa table 
    de transition suivants : 

    \begin{center}
        \begin{minipage}{0.45\textwidth} % Colonne pour le dessin
            \begin{tikzpicture}[shorten >=1pt, node distance=3cm, on grid, auto]
    
                % Définitions des styles d'états
                \node[state, initial] (1) {1};
                \node[state, accepting, right=of 1] (2) {2};
                \node[state, initial, below=of 1] (3) {3};
                \node[state, right=of 3] (4) {4};
            
                % Définition des transitions
                \path[->]
                (1) edge[above] node{$a$} (2)
                (1) edge[left] node{$a$} (3)
                (2) edge[right] node{$a$} (4)
                (3) edge[below] node{$b$} (4)
                (4) edge[sloped, below left] node{$b$} (1)
                (4) edge[loop below] node{$b$} (4)
                (2) edge[loop above] node{$b$} (2)
                (3) edge[loop below] node{$a$} (3);
            
            \end{tikzpicture}
        \end{minipage}%
        \hfill 
        \begin{minipage}{0.45\textwidth} % Colonne pour le tableau
            \begin{tabular}{c|c|c}
                T & a & b \\ \hline 
                $1$ & $\{2,3\}$ & X \\ \hline 
                \textcircled{2} & $4$ & $2$ \\ \hline 
                $3$ & $3$ & $4$ \\ \hline 
                $4$ & X & $\{1,4\}$ 
            \end{tabular}
        \end{minipage}
    \end{center}

    Déterminisons cet automate. Pour cela, nous allons renommer tous les états de l'automate en prenant en 
    compte les ensembles. 
    L'algorithme consiste donc à construire la table de transition du nouvel automate. 
    Pour chaque itération (i.e ajout d'une ligne dans la table), on effectue un parcours en largeur du nouvel automate 
    pour "découvrir" de nouveau état. On créé ainsi un "automate des parties". 

    \begin{figure}[h]
    \begin{center}
        \begin{minipage}{0.45\textwidth}
            \begin{tabular}{c|c|c}
                & a & b \\ \hline 
                $I = \{1,3\}$ & $II$ & $III$ \\ 
                $II = \{2,3\}$ & $IV$ & $V$ \\ 
                $III = \{4\}$ & - & $VI$ \\ 
                $IV = \{3,4\}$ & $VII$ & $VI$ \\ 
                $V = \{2,4\}$ & $III$ & $ VIII$ \\ 
                $VI = \{1,4\}$ & $ II$ & $VI$ \\ 
                $VII = \{3\}$ & $VII$ & $III$ \\ 
                $VIII = \{1,2,4\}$ & $IX$ & $VIII$ \\ 
                $IX = \{2,3,4\}$ & $IV$ & $VIII$ 
            \end{tabular}
            \caption{Table de l'AFD}
        \end{minipage}
        \hfill 
        \begin{minipage}{0.45\textwidth}
            \begin{tikzpicture}[scale=0.8, transform shape, shorten >=1pt, node distance=3cm, on grid, auto]
                % 1er niveau
                \node[state] (I) {I};

                % 2nd niveau
                \node[state] (II) [below left=2cm and 2cm of I] {II};
                \node[state] (III) [below right=2cm and 2cm of I] {III};

                % 3eme niveau
                \node[state] (IV) [below left=2cm and 2cm of II] {IV};
                \node[state] (V) [below right=2cm and 2cm of II] {V};
                \node[state] (VI) [below=2cm of III] {VI};

                % 4eme niveau
                \node[state] (VII) [below=2cm of IV] {VII};
                \node[state] (VIII) [below=2cm of V] {VIII}; 

                % 5eme niveau
                \node[state] (IX) [below=2cm of VIII] {IX};
                
                % Branches 
                \path[-]
                (I) edge[left] node{a} (II)
                (I) edge[left] node{b} (III)
                (II) edge[left] node{a} (IV)
                (II) edge[left] node{b} (V)
                (III) edge[left] node{b} (VI)
                (IV) edge[left] node{} (VII)
                (V) edge[left] node{} (VIII)
                (VIII) edge[left] node{} (IX);
            \end{tikzpicture}
            \caption{Automate des parties}
        \end{minipage}
    \end{center} 
    \end{figure}
\end{example}








\chapter{Automate Fini A Epsilon Transition}

\minitoc % Ajoute le sommaire local ici

\setlength{\parindent}{0pt}
\renewcommand{\labelitemi}{\textbullet} % Utiliser des points noirs (•)


% ==================================================================================================================================
% Introduction 

Définissons un nouveau type d'automates non déterministes. Les automates non déterministes à $\varepsilon$-transition. 
Il diffèrent des premiers puisque l'on va permettre le changement d'état sans lecture de lettres lors 
de la lecture d'un mot. Pour cela, nous allons définir une transition $\varepsilon$. 
Cela peut se voir comme une transition via le mot vide. 

% ==================================================================================================================================
% Définitions et propriétés

\section{Généralités}

\begin{definition}[Automate Fini à $\varepsilon$-transitions (AFN$\varepsilon$)]
    Un automate fini à $\varepsilon$-transitions est un quintuplet: 
        \[ \mathcal{A} = (Q,\Sigma, T, I, A) \] 
    définit de la même façon que les automates précédents mais où: 
        \[ T : Q \times \Sigma \cup \{\varepsilon\} \longrightarrow \mathcal{P}(Q) \] 
\end{definition}

Ici, le changement spontané d'état sans lecture de lettre sera donc caractérisé par une nouvelle 
entrée dans la table de transition $\varepsilon$. 

\begin{example}
    Soit le langage $L := a^*b^*$. Un automate reconnaissant ce langage peut être écrit avec une 
    $\varepsilon$-transition. Ecrivons aussi sa table de transition: 
    \begin{center}
        \begin{minipage}{0.45\textwidth} % Colonne pour le dessin
            \begin{tikzpicture}[shorten >=1pt, node distance=3cm, on grid, auto]
    
                % Définitions des styles d'états
                \node[state, initial] (1) {1};
                \node[state, accepting, right=of 1] (2) {2};
            
                % Définition des transitions
                \path[->]
                (1) edge[above] node{$\varepsilon$} (2)
                (1) edge[loop below] node{$a$} (1)
                (2) edge[loop below] node{$b$} (2); 

            \end{tikzpicture}
        \end{minipage}%
        \hfill 
        \begin{minipage}{0.45\textwidth} % Colonne pour le tableau
            \begin{tabular}{c|c|c|c}
                T & a & b & $\varepsilon$ \\ \hline 
                1 & 1 & - & 2 \\ \hline 
                \textcircled{2} & - & 2 & - \\ \hline 
            \end{tabular}
        \end{minipage}
    \end{center}
\end{example}

Lors de la lecture d'un mot, les transitions peuvent être très aléatoires en fonction du nombre 
d'$\varepsilon$-transitions possibles de l'état courant. Un tel automate est donc hautement non déterministe. 

\newpage 

\begin{definition}[Lecture d'un mot]
    Soit $w = l_1 l_2 \dots l_n $ un mot sur $\Sigma$. 
    Soit $w' = a_1 a_2 \dots a_p$ le mot $w$ $\varepsilon$-complété (rembourré par des $\varepsilon$) tel que :
    \begin{itemize}
        \item $p \geqslant n$ 
        \item $ \forall i \in \llbracket 1, n \rrbracket, \; a_i \in \{l_1, \dots, l_n\} \cup \{\varepsilon\}$ 
        \item $l_1 l_2 \dots l_n = a_1 a_2 \dots a_p$ (du point de vue du produit de concaténation)
    \end{itemize}
    Une lecture du mot $w$ par $ \mathcal{A}$ est une lecture par $ \mathcal{A}$ de n'importe quel $w'$, 
    un $\varepsilon$-complété de $w$. 
\end{definition}

\begin{remark}
    Tout comme pour les automates précédants, un mot appartient au langage d'un automate ssi 
    la lecture de ce mot par celui-ci se finit sur au moins un état acceptant de l'automate. 

    \vspace{0.3cm}

    La lecture d'un mot par un AFN$\varepsilon$ conduira donc à la construction d'une forêt de lecture de ce mot 
    par l'automate. 
\end{remark}

\begin{example}[Lecture d'un mot]
    Soit l'automate fini non déterministe à $\varepsilon$-transitions précédent. 
    Soit le mot $abb$. Construisons la forêt de lecture de $abb$ par $ \mathcal{A}$. 

    \begin{center}
        \begin{tikzpicture}[shorten >=1pt, node distance=3cm, on grid, auto, scale=0.7]
    
            % Définitions des styles d'états
            \node[state, initial] (1) {1};
            \node[state, accepting, below=of 1] (2) {2};
            \node[state, right=of 1] (11) {1};
            \node[state, right=of 2, xshift=-1.1cm] (X1) {X};
            \node[state, right=of 11] (X2) {X};
            \node[state, below=of 11] (22) {2};
            \node[state, right=of 22] (222) {2};
            \node[state, accepting, right=of 222] (2222) {2};
        
            % Définition des transitions
            \path[->]
            (1) edge[above] node{$a$} (11)
            (1) edge[right] node{$\varepsilon$} (2)
            (2) edge[above] node{} (X1)
            (11) edge[above] node{} (X2)
            (11) edge[right] node{$\varepsilon$} (22)
            (22) edge[above] node{$b$} (222)
            (222) edge[above] node{$b$} (2222);
    
        \end{tikzpicture}
    \end{center}

    Il existe un chemin d'un racine vers une feuille acceptante dans la forêt de lecture: 
        \[ 1 \overset{a}{\longrightarrow} 1 \overset{\varepsilon}{\longrightarrow} 2 \overset{b}{\longrightarrow} 2 \overset{b}{\longrightarrow} 2 \] 
    Donc par définition, $abb \in L( \mathcal{A} )$. 
\end{example}

% ==================================================================================================================================
% Déterminisation d'un AFNe

\section{Déterministation}

Dans le chapitre précédent, un théorème nous permet de dire que pour tout automate fini non déterministe, 
il existe un automate fini déterministe équivalent. Ainsi, à chaque fois que l'on considère un AFN$\varepsilon$, 
on est sûr qu'il existe un AFD équivalent. 

\vspace{0.3cm}

Pour la déterminisation d'un AFN$\varepsilon$, nous allons avoir besoin d'une définition supplémentaire... 

\begin{definition}[Clôture]
    Soit $\mathcal{A} = (Q,\Sigma, T, I, A)$ un AFN$\varepsilon$. 
    Pour tout $q \in Q$, on appelle clôture de $q$ l'ensemble des états accessibles à partir de $q$ sans lecture de 
    lettre lors de la lecture d'un mot dans l'automate. 

    Autrement dit, la clôture de $q$ est l'ensemble des états accessibles depuis $q$ dans le sous-graphe de $ \mathcal{A}$
    restreint aux $\varepsilon$-transitions. 
    
    On note $cl(q)$ la clôture de $q$. 
\end{definition}


L'idée de la déterministation d'un AFN$\varepsilon$ est, non plus de regrouper des états, mais d'étendre les transitions 
de l'automate à tous les états accessibles par $\varepsilon$-transitions. 

De plus, si un état acceptant est accessible uniquement par $\varepsilon$-transtion depuis un état, alors celui-ci 
hérite du caractère acceptant de l'état atteint. 

\begin{proposition}[Algorithme de Déterministation]
    Soit $\mathcal{A} = (Q,\Sigma, T, I, A)$ un AFN$\varepsilon$. On construit l'automate fini déterministe $ \mathcal{A}'$ 
    équivalent à $ \mathcal{A}$ en :
    \begin{enumerate}
        \item Calculer les clôtures de $ \mathcal{A}$ 
        \item Héritage : Tous les états dont la clôture contient un état acceptant sont acceptants. 
        \item Calculer les transitions étendues 
        \item Déterminisation de $ \mathcal{A'}$ par l'automate des parties (voir chap précédent)
    \end{enumerate}
\end{proposition}

\begin{example}[Déterminisation]
    Soit $ \mathcal{A}$ l'AFN$\varepsilon$ suivant : 
    
    \begin{center}
        \begin{minipage}{0.7\textwidth} % Colonne pour le dessin
            \begin{tikzpicture}[shorten >=1pt, node distance=3cm, on grid, auto]
    
                % Définitions des styles d'états
                \node[state, initial] (1) {1};
                \node[state, right=of 1] (2) {2};
                \node[state, accepting, above right=of 2] (3) {3}; 
                \node[state, below=of 3] (4) {4};
                
                % Définition des transitions
                \path[->]
                (1) edge[above] node{$a$} (2)
                (2) edge[bend left, above] node{$\varepsilon$} (3)
                (3) edge[bend left, above] node{$b$} (2) 
                (3) edge[right] node{$\varepsilon$} (4) 
                (4) edge[above] node{$a$} (2) 
                (4) edge[loop below] node{$b$} (4);

            \end{tikzpicture}
        \end{minipage}%
        \hfill 
        \begin{minipage}{0.25\textwidth} % Colonne pour le tableau
            \begin{tabular}{c|c|c|c}
                T & a & b & $\varepsilon$ \\ \hline 
                1 & 2 & - & - \\ \hline 
                2 & - & - & 3 \\ \hline 
                \textcircled{3} & - & 2 & 4 \\ \hline 
                4 & 2 & 4 & - 
            \end{tabular}
        \end{minipage}
    \end{center}

    \begin{enumerate}
        \item \textbf{Calcul des clôtures et héritage} 
            \begin{center}
                \begin{tabular}{c|c|c|c|c}
                    T & a & b & $\varepsilon$ & $cl(q)$ \\ \hline 
                    1 & 2 & - & - & $\{1\}$ \\ \hline 
                    \textcircled{2} & - & - & 3 & $\{2,3,4\} $ \\ \hline 
                    \textcircled{3} & - & 2 & 4 & $\{3,4\}$ \\ \hline 
                    4 & 2 & 4 & - & $\{4\}$
                \end{tabular}
            \end{center}
        \item \textbf{Calcul des transitions étendues}
            \begin{center}
                \begin{tabular}{c|c|c|c}
                    $\overset{\sim}{T}$ & a & b & $cl(q)$ \\ \hline 
                    1 & 2 & - & \\ \hline 
                    \textcircled{2} & 2 & $\{2,4\}$ & $\{2,3,4\}$ \\ \hline 
                    \textcircled{3} & 2 & $\{2,4\}$ & $\{3,4\}$ \\ \hline 
                    4 & 2 & 4 & \\
                \end{tabular}
            \end{center}
        \item \textbf{Déterminisation par l'automate des parties}
            \begin{center}
                \begin{minipage}{0.45\textwidth}
                    \begin{tabular}{c|c|c}
                        & a & b \\ \hline 
                        I $= \{1\}$ & II & X \\ \hline 
                        II $= \{2\}$ & II & III \\ \hline 
                        III $= \{2,4\}$ & II & III 
                    \end{tabular}
                \end{minipage}
                \hfill 
                \begin{minipage}{0.45\textwidth} % Colonne pour le dessin
                    \begin{tikzpicture}[shorten >=1pt, node distance=2cm, on grid, auto]
            
                        % Définitions des styles d'états
                        \node[state, initial] (1) {1};
                        \node[state, accepting, right=of 1] (2) {2};
                        \node[state, accepting, below=of 2] (3) {3}; 
                        
                        % Définition des transitions
                        \path[->]
                        (1) edge[above] node{$a$} (2)
                        (2) edge[loop above] node{$a$} (2) 
                        (2) edge[bend left, left] node{$b$} (3)
                        (3) edge[bend left, left] node{$a$} (2) 
                        (3) edge[loop below] node{$b$} (3);
        
                    \end{tikzpicture}
                \end{minipage}
            \end{center}
    \end{enumerate}
\end{example}

La déterminisation d'un AFN$\varepsilon$ nous permet donc de passer de la lecture d'un mot de complexité 
exponentielle (voire infinie) à un automate permettant de lire tous les mots avec une complexité linéaire. 


\chapter{Opérations entre automates}

\minitoc % Ajoute le sommaire local ici

\setlength{\parindent}{0pt}
\renewcommand{\labelitemi}{\textbullet} % Utiliser des points noirs (•)


% ==================================================================================================================================
% Introduction 

Les chapitres précédents nous ont montré que les $\varepsilon$-transitions et les transitions multiples 
nous permettent de modéliser plus facilement des langages complexes. 
Nous allons donc définir des opérations sur les automates, analogues à celles sur les langages. 
Elles nous permettront, à partir de la construction d'un langage par opérations, de construire son automate par 
opérations aussi. 

\vspace{0.3cm}

\textbf{Problématique : } Soient $L_1$ et $L_2$ deux langages reconnus respectivement par $ \mathcal{A}_2$ et $ \mathcal{A}_2$. 
Soit $T$ une opération entre langages. Comment construire un automate reconnaissant $L_1 T L_2$ ? 

% ==================================================================================================================================
% Langage Elémentaires 

\section{Langages Elémentaires}

Pour définir des opérations sur les automates, et ainsi, construire un automate par opérations pour un langage 
lui-même construit par opérations, nous allons avoir besoin de définir des langages élémentaires. 
Ces langages seront reconnus par des automates fixés, que nous connaissons d'avance. Nous en choisissons un nombre fini 
pour pouvoir les stocker en mémoire. Ils vont représenter les briques de base nous permettant de construire des automates plus 
complexes par la suite. 

\begin{proposition}[Langages Elémentaires]
    Soit $\Sigma = \{a,b\}$ un alphabet. On définit les langages élémentaires de $\Sigma$ comme :
    \begin{itemize}
        \item $L = \emptyset$ reconnu par l'automate : 
            \begin{center}
                \begin{tikzpicture}[shorten >=1pt, node distance=3cm, on grid, auto]
                    \node[state, initial] (1) {$1$};
                \end{tikzpicture}
            \end{center}
        \item $L = \{\varepsilon\}$ reconnu par l'automate :
            \begin{center}
                \begin{tikzpicture}[shorten >=1pt, node distance=3cm, on grid, auto]
                    \node[state, initial, accepting] (1) {$1$};
                \end{tikzpicture}
            \end{center}
        \item $L = \{a\}$ reconnu par le langage :
            \begin{center}
                \begin{tikzpicture}[shorten >=1pt, node distance=3cm, on grid, auto]
                    \node[state, initial] (1) {$1$};
                    \node[state, accepting, right of =1] (2) {$2$}; 
                    \path[->]
                    (1) edge[above] node{$a$} (2);
                \end{tikzpicture}
            \end{center}
        \item $L = \{a_1 \dots a_n\}$ reconnu par le langage :
            \begin{center}
                \begin{tikzpicture}[shorten >=1pt, node distance=3cm, on grid, auto]
                    \node[state, initial] (1) {$1$};
                    \node[state, right of =1] (2) {$2$}; 
                    \node[state, right of=2] (3) {$a_n$};
                    \node[state, accepting, right of=3] (4) {$n$};
                    \path[->]
                    (1) edge[above] node{$a_1$} (2)
                    (2) edge[above] node{$a_2$} (3)
                    (3) edge[above] node{$\dots$} (4);
                \end{tikzpicture}
            \end{center}
    \end{itemize}
\end{proposition}

% ==================================================================================================================================
% Langage complémentaire

\section{Automate Complémentaire}

A partir d'un langage, on peut définir son langage complémentaire. De même, on peut définir l'automate complémantaire 
reconnaissant ce langage. 

\begin{definition}[Automate Complémentaire]
    Soit $L$ un langage reconnu un automate $ \mathcal{A} = (Q,\Sigma,T,q_0,A)$ \textbf{complet}. 
    L'automate reconnaissant le langage complémentaire de $L$ est $ \overline{ \mathcal{A}} = (Q,\Sigma,T,q_0,Q \backslash A)$. 
\end{definition}

\begin{remark}
    Attention, pouvoir "passer au complémentaire" pour un automate, il faut que celui-ci soit complet. 
\end{remark}

% ==================================================================================================================================
% Somme d'automates

\section{Somme d'automates}

\begin{definition}[Somme d'automates]
    Soient $L_1$ et $L_2$ deux langages respectivement reconnus par $ \mathcal{A}_1$ et $ \mathcal{A}_2$. 
    L'automate reconnaissant $L_1 + L_2$ est l'automate fini non déterministe construit par \textbf{l'union disjointe}
    de $ \mathcal{A}_1$ et $ \mathcal{ A}_2$. On l'appelle \textbf{automate somme} des automates $ \mathcal{A}_1$ 
    et $ \mathcal{A}_2$. 
\end{definition}

Cet automate n'est pas déterministe puisqu'il contient deux états initiaux mais que l'on peut déterminiser. 

\begin{example}
    Soient les langages suivants sur $\Sigma = \{a,b\}$ :
    \begin{align*}
        L_1 &= \{ \text{mots de } \Sigma \text{ terminant par } a \} \\ 
        L_2 &= \{ \text{mots de } \Sigma \text{contenant un nombre pair de } a \}
    \end{align*}

    Ces langages sont reconnus par les automates suivants :

    \begin{center}
        \begin{minipage}{0.45\textwidth}
            \begin{tikzpicture}[shorten >=1pt, node distance=3cm, on grid, auto]
                \node[state, initial] (1) {$1$};
                \node[state, right of=1, accepting] (2) {$2$};
                
                \path[->]
                (1) edge[above, bend right] node{$a$} (2)
                (2) edge[above, bend right] node{$b$} (1)
                (2) edge[loop right] node{$a$} (2)
                (1) edge[loop above] node{$b$} (1); 
            \end{tikzpicture}
        \end{minipage}%
        \hfill 
        \begin{minipage}{0.45\textwidth}
            \begin{tikzpicture}[shorten >=1pt, node distance=3cm, on grid, auto]
                \node[state, initial] (1) {$1$};
                \node[state, right of=1, accepting] (2) {$2$};
                
                \path[->]
                (1) edge[above, bend right] node{$a$} (2)
                (2) edge[above, bend right] node{$a$} (1)
                (2) edge[loop right] node{$b$} (2)
                (1) edge[loop above] node{$b$} (1); 
            \end{tikzpicture}
        \end{minipage}
    \end{center}

    On construit l'automate $ \mathcal{A}$ comme la juxtaposition disjointe des deux automates :
    \begin{center}
        \begin{minipage}{0.45\textwidth}
            \begin{tikzpicture}[shorten >=1pt, node distance=3cm, on grid, auto]
                \node[state, initial] (1) {$1$};
                \node[state, right of=1, accepting] (2) {$2$};
                
                \path[->]
                (1) edge[above, bend right] node{$a$} (2)
                (2) edge[above, bend right] node{$b$} (1)
                (2) edge[loop right] node{$a$} (2)
                (1) edge[loop above] node{$b$} (1); 
            \end{tikzpicture}
        \end{minipage}%
        \hfill 
        \begin{minipage}{0.45\textwidth}
            \begin{tikzpicture}[shorten >=1pt, node distance=3cm, on grid, auto]
                \node[state, initial] (3) {$3$};
                \node[state, right of=3, accepting] (4) {$4$};
                
                \path[->]
                (3) edge[above, bend right] node{$a$} (4)
                (4) edge[above, bend right] node{$a$} (1)
                (4) edge[loop right] node{$b$} (4)
                (3) edge[loop above] node{$b$} (3); 
            \end{tikzpicture}
        \end{minipage}
    \end{center}

    Que l'on doit ensuite déterminiser... 
\end{example}

Les définitions de langages élémentaires nous permettent de savoir que tout langage réduit à un mot est automatique. 
On en déduit le théorème suivant : 

\begin{theorem}[Langage Fini]
    Tout langage fini est automatique. 
\end{theorem}


% ==================================================================================================================================
% Intersection d'automates

\section{Intersection d'Automates}

\begin{proposition}
    Soient $L_1$ et $L_2$ reconnus par les automates fini déterministes suivants $ \mathcal{A}_1$ et $ \mathcal{A}_2$. 
    On a alors : 
        \[ \overline{L_1 \cap L_2} = \overline{L_1} + \overline{L_2} \]
    On peut donc en conclure que : 
        \[ \boxed{L_1 \cap L_2 = \overline{\overline{L_1} + \overline{L_2}}} \]  
    Ainsi, l'intersection de deux automates peut être construite par somme et complémentarité. 
\end{proposition}

En pratique nous utuliseront plutôt l'automate des couples : 

\begin{definition}[Automate des Couples]
    Soient $ \mathcal{A}_1 = \{Q_1, \Sigma, T_1, Q_0, A_1\}$ reconnaissant le langage $L_1$ et 
    $ \mathcal{A}_2 = \{Q_2, \Sigma, T_2, q_0', A_2\}$ reconnaissant le langage $L_2$. On définit 
    l'automate des couples : 
        \[ A = \{Q_1 \times Q_2, \Sigma, T, (q_0,q_0'), A_1 \times A_2\} \] 
    reconnaissant le langage $L_1 \cap L_2$ où 
        \[ \forall i \in \Sigma, \forall q_1, q_2 \in Q_1, \forall q_1', q_2' \in Q_2 \text{ tels que } q_2 = T_1(q_1,j) \text{ et } q_2' = T(q_1',i) \]
        \[ \text{alors } T'((q_1, q_1'), i) = (q_2, q_2') \] 
\end{definition}

\begin{proposition}
    Si $ \mathcal{A}_1$ possède $n \in \N$ états et que $ \mathcal{A}_2$ possède $p \in \N$ états, alors 
    $ \mathcal{A}_1 \cap \mathcal{A}_2$ possède $n \times p$ états. 
    Pour de gros automates, cette méthode peut donc engendrer des très gros. 
    Même si la façon de les construire est assez simple et ressemble beaucoup à la déterminisation. 

    L'automate obtenu est, de plus, déterministe. 
\end{proposition}


% ==================================================================================================================================
% Différences d'automates

\section{Différence d'Automates}

\begin{proposition}
    Soient $ \mathcal{A}_1$ et  $ \mathcal{A}_2$ deux automates reconnaissant respectivement les langages $L_1$ et $L_2$. 
    Pour reconnaître le langages $L_1 \backslash L_2$, on peut simplement construire l'automate reconnaissant :
        \[ L_1 \backslash L_2 = L_1 \cap \overline{L_2} \] 
\end{proposition}

% ==================================================================================================================================
% Langages Automatiques 

\newpage

\section{Langages Automatiques}

Essayons maintenant de déduire des conditions nécessaire pour qu'un langage soit automatique. 
D'après ce que l'on a vu grâce aux opérations, on peut déjà énoncer la proposition suivante : 
\begin{proposition}
    Les langages réguliers sont tous automatiques. 
\end{proposition}

Proposition que nous élargirons plus tard grâce au théorème de Kleene. 

\subsection{Théorème de pompage}

\begin{theorem}[Pompage (faible)]
    Soit $L$ un langage. Supposons $L$ automatique. Soit $w \in L$ alors pour toute décomposition de $w$ de la forme :
        \[\exists w_1, w_2, w_3 \in L, w = w_1 w_2 w_3 \]
    \underline{alors} cette décomposition est gonflable. 
        \[ \text{ i.e } \forall k \in \N, w_1 w_2 ^k w_3 \in L \]
\end{theorem} 

Ce théorème n'est pas idéal pour montrer qu'un langage est automatique. En revanche, sa contraposé de la forme :
\begin{quote}
    \textbf{Non Pompable $ \Longrightarrow $ Non Automatique}
\end{quote}

Est en pratique très utilisée pour montrer qu'un langage n'est pas automatique. 
\chapter{Langage d'un Automate}

\minitoc % Ajoute le sommaire local ici

\setlength{\parindent}{0pt}
\renewcommand{\labelitemi}{\textbullet} % Utiliser des points noirs (•)


% ==================================================================================================================================
% Introduction 

Maintenant que nous savons bien manipuler les automates fini déterministes et non déterministes, il serait utile, 
à partir d'un automate de pouvoir déterminer le langage qu'il reconnaît. Pour cela nous auront besoin de définir 
les systèmes d'équations au langages, d'introduire le Lemme d'Arden. Nous finirons par énoncer le théorème de Kleene 
caractérisant les langages automatiques. 

% ==================================================================================================================================
% Langage d'un automate fini

\section{Langage d'un automate fini}

\begin{definition}[Langage d'arrivée]
    Soit $ \mathcal{A} = \{Q, \Sigma, T, q_0, A\}$ un automate fini. On définit le langage d'arrivée à l'état $ q \in Q$, 
    noté $ L_q$ l'ensemble des mots de $ \Sigma^*$ dont la lecture par $ \mathcal{A}$ débute par $q_0$ et finit en $q$. 
\end{definition}

On peut alors définir la langage d'un automate comme :

\begin{proposition}
    Soit $ \mathcal{A} = \{Q, \Sigma, T, q_0, A\}$ un automate fini. Soit $\{L_q \; | \; i \in A \}$ 
    l'ensemble des langages d'arrivés aux états acceptants de l'automate $ \mathcal{A}$. On a alors l'égalité suivante : 
        \[ L( \mathcal{A}) = \sum_{q \in A} L_q \]
    Autrement dit, le langage de $ \mathcal{A}$ est la somme de tous ses langages d'arrivé aux états acceptants. 
\end{proposition}

\begin{definition}[Système d'équations aux langages]
    Soient un ensemble $X_1, \dots, X_n$ de langages sur un même alphabet $\Sigma$. 
    Un système d'équations aux langages est un ensemble d'équations de la forme : 
        \[ X_i = \sum_{j=1}^{n} a_{ij} X_j + b_j \quad \forall i \in \llbracket 1, n \rrbracket \] 
    où $ \forall i,j \in \llbracket 1, n \rrbracket, a_{ij} \in \Sigma, b_i \in \Sigma^*$
\end{definition}

\begin{proposition}
    Soit $ \mathcal{A} = \{Q, \Sigma, T, q_0, A\}$ un automate fini.
    A partir des définitions, on peut donc représenter le langage d'un automate par un système d'équations aux langages. 
    Elles associent à chaque $L_q$ une équation de langages dérivant de l'automate. 
    
    \vspace{0.2cm}
    
    Ainsi si un état $q$ a des transitions vers d'autres états selon les lettres $a \in \Sigma$ et si $q$ est un état 
    acceptant alors l'équation associée à $L_q$ est de la forme 
        \[ 
            \begin{cases}
                L_q = \sum_{(q,a,q') \in T} a L_{q'} + \{\varepsilon\}  \quad \text{si q est un état initial} \\ 
                L_q = \sum_{(q,a,q') \in T} a L_{q'} \quad \text{sinon}
            \end{cases}
        \] 
\end{proposition}

Il nous faut maintenant être capable de résoudre ces équations pour déterminer les $L_q$ et ainsi le langage de l'automate. 


% ==================================================================================================================================
% Lemme d'Arden

\section{Lemme d'Arden}

Le Lemme d'Arden permet de résoudre de telles équations. Il fut démontré en 1961 par Dean N. Arden. Voici son énoncé : 

\begin{lemma}[Arden]
    Soient $A$ et $B$ deux langages. Le langage 
        \[ \boxed{ L = A^*B } \]
    est le plus petit langage (pour l'inclusion ensembliste) qui est solution de l'équation 
        \[ \boxed{ (E) : X = (AX) \cup B } \] 
    De plus, si $A$ ne contient pas le mot vide $\varepsilon$, $A^*B$ est l'unique solution de cette équation. 
\end{lemma}

Ainsi, puisque chacune des équations aux langages vues précédement sont de la forme $ L = AL + B$, on peut donc 
résoudre de tels systèmes et calculer explicitement le langage d'un automate. 

\begin{example}

    Déterminons le langage de l'automate suivant :

    \begin{center}
        \begin{tikzpicture}[shorten >=1pt, node distance=3cm, on grid, auto]
            \node[state, initial] (1) {$1$};
            \node[state, accepting, right of=1] (2) {$1$};
            \node[state, accepting, right of=2] (3) {$3$};

            \path[->]
            (1) edge[loop below] node{$a$} (1)
            (1) edge[above] node{$b$} (2)
            (2) edge[loop below] node{$a$} (2) 
            (2) edge[above] node{$b$} (3) 
            (3) edge[loop below] node{$a$} (3);
        \end{tikzpicture}
    \end{center}

    Les langages d'arrivés vérifient donc le système d'équations aux langages suivant :
    \[ 
        \begin{cases}
            L_1 = L_1 a + \varepsilon \\ 
            L_2 = L_1 b + L_2 a \\ 
            L_3 = L_2 b + L_3 b 
        \end{cases}
    \] 
    Que l'on résout progressivement grâce au Lemme d'Arden : 
    \[ 
        \begin{cases}
            L_1 = \varepsilon a^* = a^* \\ 
            L_2 = L_2 a + a^* b \\ 
            L_3 = L_3 a + L_2 b 
        \end{cases}
        \quad \iff \quad 
        \begin{cases}
            L_1 = a^* \\ 
            L_2 = a^*ba^* \\ 
            L_3 = L_3a + a^*ba^* 
        \end{cases} 
    \quad \iff \quad 
        \begin{cases}
            L_1 = a^* \\ 
            L_2 = a^*ba^* \\ 
            L_3 = a^*ba^* b a^* \\ 
        \end{cases}
    \]

    D'où : 
        \begin{align*}
            L( \mathcal{A}) &= a^*ba^* + a^*ba^* b a^* \\ 
            &= a^*ba^*(\varepsilon + ba^*) 
        \end{align*}
    Donc $ \mathcal{A}$ reconnaît les mots qui contiennent 1 ou 2 b. 
\end{example}

Cet algorithme de résolution est très utile en terme pratique mais il apporte aussi une précisions supplémentaire sur 
les langages automatiques. En effet, à partir d'un automate (i.e langage automatique), on peut écrire le langage 
reconnu comme une expression régulière. D'où le résultat suivant : 

\begin{prop}[Langages Automatiques]
    Tout langage automatique peut être décrit par une expression régulière. 
\end{prop}

On en déduit donc le théorème de Kleene établissant définitivement le lien entre les langages réguliers et automatique : 

\begin{theorem}[Kleene]
    Les langages réguliers sont les mêmes que les langages automatiques. 
\end{theorem}





\chapter{Langages Algébriques et Automates à Piles}

\minitoc % Ajoute le sommaire local ici

\setlength{\parindent}{0pt}
\renewcommand{\labelitemi}{\textbullet} % Utiliser des points noirs (•)

% ==================================================================================================================================
% Introduction 

Précédement, nous avons vu que les langages automatiques sont de très bonnes propriétés. Ils sont stables 
pour la plupart des opérations définies sur les langages. De plus, le théorème de Kleene nous a permis d'établir 
le lien direct entre langages automatiques et langages réguliers. 
La simplicité de la représentation sagitale des automates permet de les implémenter facilement algorithmiquement. 
Il sont, de plus, facile à manipuler à la main et permettent de rapidement "voir" les langages reconnus. 

Cependant, cette simplicité a un certain coût, celui de ne pas pouvoir reconnaître des langages "compliqués", notamment
ceux où il faut "compter" les lettres. Un automate fini déterministe ne peut donc pas reconnaître le langage composé 
d'autant de $a$ que de $b$. 
On va donc chercher à indroduire une nouvelle théorie, celle des \textbf{grammaires formelles} qui nous permettra
de reconnaître de tels langages. 


% ==================================================================================================================================
% Grammaires Formelles

\section{Grammaires Formelles}

\subsection{Contexte et définition}

Les grammaires formelles ont initialement été développés par des linguistes, notamment Noam Chomsky en 1955. 
L'objectif était de développer un méthode systématique de traduction entre différentes langues. 
Ils se sont alors heurtés au problème des mêmes mots qui admettent plusieurs traductions en fonction du contexte de la 
phrase et n'ont pas pu aboutir leur oeuvre. 

\vspace{0.3cm}

Or en informatique, pour l'étude de la syntaxe de langages de programmation, le problème du contexte ne se pose pas. 
Leur théorie a donc été récupérée pour la vérification syntaxique. 

\vspace{0.3cm}

L'idée est donc de représenter un langage \textbf{récursivement} par un ensemble de règles de production composées 
d'un axiome de départ et de différentes règles de productions ou de réécriture. 

Nous utilisons souvent cette approche pour la gestion de types en Caml en définissant tes types récurisivement. 


\begin{definition}[Grammaire Formelle]
    Une grammaire formelle est un quadruplet 
        \[ G = (\Sigma, V, S, P) \] 
    où 
    \begin{itemize}
        \item $\Sigma$ est un \textbf{alphabet terminal} dont chaque élément ne peut se réécrire plus simplement. 
        \item $V$ est \textbf{l'alphabet auxiliaire} (disjoint de $\Sigma$) composé de variables, qui ne 
        peuvent pas non plus se réécrire. 
        \item $S$ est la variable de départ, appelé axiome. 
        \item $P$ est un ensemble de règles dites \textbf{de production} ou de réécriture du type 
            \[ X \longrightarrow w \quad X \in V \text{ et } w \in \left( V \cup \Sigma \right)^* \] 
    \end{itemize}
\end{definition}

Par convention, on notera toujours les variables en majuscule et les éléments terminaux en minuscules. 
En pratique, on regroupera plusieurs réécritures d'une même variable sur la même ligne en les séparant par 
des barres verticales de la forme :
    \[ X \longrightarrow w_1 | w_2 | \dots | w_p \iff 
        \begin{cases}
            X \longrightarrow w_1 \\ 
            X \longrightarrow w_2 \\ 
            \vdots \\ 
            W \longrightarrow w_p 
        \end{cases}
    \] 


\subsection{Réécriture d'un mot et langages algébriques}

L'obectif d'une grammaire formelle, vous l'aure compris, est de réécrire un mot récursivement jusqu'à arriver à des 
éléments terminaux. 

\begin{definition}[Réécriture d'un mot]
    Soit $ G = (\Sigma, V, S, P)$ une grammaire formelle. Soient $u,v \in \left( V \cup \Sigma \right)^*$ deux mots. 
    On dit que \textbf{$u$ peut se réécrire en $v$ en une étape} et on note :
        \[ u \vdash v \] 
    si il existe des décompositions de $u$ et $v$ en 
        \[ u = u_1 X u_2 \text{ et } v = u_1 w u_2 \] 
    et que $G$ contient la règle de production :
        \[ X \longrightarrow w \] 
\end{definition}

\newpage 

Plus généralement, on peut définir la réécriture en plusieurs étapes de la forme : 

\begin{definition}[Réécriture (2)]
    Soit $ G = (\Sigma, V, S, P)$ une grammaire formelle. Soient $u,v \in \left( V \cup \Sigma \right)^*$ deux mots.
    On dit que \textbf{$u$ peut se réécrire en $v$} ou que \textbf{$v$ dérive en $u$} en un nombre quelconque de fois si 
    il existe $u_1, \dots, u_p \in \left( V \cup \Sigma \right)^*$ tels que 
        \[ u \vdash u_1 \vdash u_2 \vdash \dots \vdash u_p \vdash v \] 
    On note alors 
        \[ u \overset{*}{\vdash} v \] 
\end{definition}

On peut maintetant définir les langages engendrés par des grammaires formelles et les langages algébriques, le coeur de ce 
chapitre. 

\begin{definition}[Langage Engendré]
    La \textbf{langage engendré} par une grammaire formelle $ G = (\Sigma, V, S, P)$ est l'ensemble des mots de $\Sigma^*$ 
    qui dérivent de l'axiome $S$ en un nombre quelconque d'étapes. On le note, comme pour les automates, en $L(G)$. 
\end{definition}

\begin{definition}[Langage Algébrique]
    Un langage engendré par une grammaire est appelé \textbf{langage algébrique}. 
\end{definition}



\subsection{Arbre de dérivation d'un mot}

\begin{definition}[Arbre de dérivation d'un mot]
    Soit $ G = (\Sigma, V, S, P)$ une grammaire formelle. On appelle l'arbre de dérivation de $w \in Sigma^*$ l'arbre dont :
    \begin{itemize}
        \item La racine est $S$ 
        \item Tous les sommets intérieurs appartiennent à $V$ 
        \item Toutes les feuilles appartiennent à $\Sigma \cup \{\varepsilon\}$ 
        \item Si un sommet intérieur $X$ a pour fils $X_1, \dots, X_p$ alors la règle 
            \[ X \longrightarrow X_1 | \dots | X_p \in P \] 
        \item Le mot obtenu en visitant les feuilles de l'arbre par un parcours profondeur préfixe de l'arbre 
            est un mot de $L(G)$
    \end{itemize}
\end{definition}

\begin{definition}[Grammaire Ambiguë]
    Soit $G$ une grammaire. On dit que $G$ est ambiguë s'il existe un mot $w_ \in L(G)$ possédant
    deux arbres de dérivation différents. 
\end{definition}

En pratique une telle grammaire est pas très utilisée. En effet, en informatique, il ne serait pas très pratique 
de pouvoir compiler un code en deux expressions différentes d'un autre langage. On ne saurait pas laquelle choisir. 
Il faut que la dérivation puisse se faire de façon unique.

\begin{example}[Grammaire Formelle et Arbre de dérivation]
    Soit $ \Sigma = \{a,b\}$ . 
    Soit la grammaire formelle $G$ définie par les règles suivantes telles que $ V = \{S\}$ : 
        \[  P : 
            \begin{cases}
                S \longrightarrow aSa \\ 
                S \longrightarrow SbS \\ 
                S \longrightarrow \varepsilon
            \end{cases}
            \iff 
            \begin{cases}
                S \longrightarrow aSa \; | \; bSb \; | \;  \varepsilon
            \end{cases}
        \] 
    Soit $abaaba \in G$ on a alors l'arbre de dérivation suivant pour ce mot : 
    \begin{center}
        \begin{forest}
            [S
                [a]
                [S 
                    [b]
                    [S 
                        [a]
                        [S [$\varepsilon$]]
                        [a]
                    ]
                    [b]
                ]
                [a]
            ]
            \end{forest}
    \end{center}
    Cette grammaire reconnaît bien les palindrômes pairs. 
\end{example}

\begin{remark}
    Lors de la dérivations de mots par une grammaire, on remarque qu'il est plus facile que les règles de dérivation 
    possèdent des traces initiales ou finales uniques telles que les $a$ et les $b$. 
    Elles permettent d'identifier plus facilement les règles à utiliser pour les dérivations. 
\end{remark}



\subsection{Grammaires Régulières}

Nous allons ici faire le lien entre les deux modèles présentés précedement, les automates fini et les grammaires formelles. 
Nous allons ainsi définir les grammaires régulières qui permettent de représenter les automates fini déterministes sous 
la forme que nous venons d'introduire. 

\begin{prop}[Représentation d'un langage automatique]
    Soit $ \mathcal{A} = (Q, \Sigma, T, q_0, A)$ un automate fini déterministe. 
    Le langage $L$ reconnu par cet automate peur être engendré par la grammaire : 
        \[ G = (Q, \Sigma, q_0, P) \] 
    dont les variables auxiliaires sont les états de l'automate et où $P$ est l'ensemble 
    des productions de la forme :
        \[ q \longrightarrow x. T(q,x) \quad \text{où } q \in Q \text{ et } x \in \Sigma \] 
        \[ q \longrightarrow \varepsilon \quad \text{ si } q \in A \] 
\end{prop}

On peut donc représenter facilement n'importe quel langage automatique par une grammaire formelle. 
D'où le théorème suivant. 

\begin{theorem}[Langage Automatique et Grammaire Formelle]
    Tout langage automatique (reconnaissable par un automate fini) est algébrique (reconnaissable par une grammaire formelle). 
    
    \vspace{0.2cm}

    L'ensemble des langages automatiques est même strictement inclus dans l'ensemble des langages algébriques. 
    Autrement dit, certains langages sont reconnaissables par une grammaire formelle mais pas par un automate. 
\end{theorem}

On définit ainsi les grammaires régulières. 

\begin{definition}[Grammaire Régulière]
    Une grammaire régulière est une grammaire formelle dont toutes les règles de production de $P$ sont de la forme :
        \[ X \longrightarrow a.Y \quad \text{ou } X \longrightarrow \varepsilon \]
    où $X,Y \in V$ et $a \in \Sigma$.  
\end{definition}

Une grammaire régulière est donc conçue de façon à "laisser des traces" explicites de la structure des 
mots pour faciliter les dérivations. De même que précédement, on peut passer d'une grammaire régulière à un automate 
fini déterministe. 

\begin{proposition}[Représentation d'une grammaire régulière]
    Soit $G$ une grammaire régulière. Soit $L$ le langage reconnu par $G$.
    L'automate fini déterministe reconnaissante aussi $L$ est : 
        \[ \mathcal{A} = (V,\Sigma, T, q_0 = S, A) \] 
    dont les états sont les variables auxiliaires de $G$ et dont les transitions sont définies par : 
        \[ q' = T(q,x) \quad \text{si} \quad q \longrightarrow x q' \in P \] 
    et dont les états acceptants sont définis par :
        \[ q \in A \quad \text{si} \quad q \longrightarrow \varepsilon \in P \]  
\end{proposition}

\begin{remark}
    Grâce à cette propriété, les langages automatiques (réguliers) sont donc exactement les langages reconnus par des grammaires régulières. 
    D'où le nom...
\end{remark}

\begin{example}[Construction d'une grammaire régulière]
    Définissons le langage $L$ reconnaissant les mots contenant un nombre pair de $a$ et impair de $b$. 
    Alors ce langage est reconnu par l'AFn suivant : 
    \begin{center}
        \begin{tikzpicture}[shorten >=1pt, node distance=3cm, on grid, auto]
            \node[state, initial] (1) {$1$};
            \node[state, accepting, right of=1] (2) {$2$};
            \node[state, below of=1] (3) {$3$};
            \node[state, below of=2] (4) {$4$};

            \path[->]
            (1) edge[above, bend right] node{$b$} (2)
            (2) edge[above, bend right] node{$b$} (1)
            (2) edge[right, bend right] node{$a$} (4) 
            (4) edge[right, bend right] node{$a$} (2) 
            (1) edge[right, bend right] node{$a$} (3)
            (3) edge[right, bend right] node{$a$} (1)
            (3) edge[above, bend right] node{$b$} (4)
            (4) edge[above, bend right] node{$b$} (3);
        \end{tikzpicture}
    \end{center}
    
    D'après la propriété précédente, $L$ est reconnu par la grammaire régulière $G = (\Sigma, V, S, P)$
    où $V = \{S,A,B,C\}$ et les états sont représentés par :
        \[ 
            \begin{cases}
                1 \longrightarrow S \\ 
                2 \longrightarrow A \\ 
                3 \longrightarrow B \\ 
                4 \longrightarrow C
            \end{cases} \] 
    On peut ensuite déterminer les règles de production à partir du voisinnage sortant 
    de chaque état de l'automate. De plus, puisque $A$ est un état acceptant, on y rajouter $\varepsilon$. 
    \[ P : 
            \begin{cases}
                S \longrightarrow bA \; | \; aB \\ 
                A \longrightarrow bS \; | \; aC \; | \; \varepsilon \\ 
                B \longrightarrow aS \; | \; cB \\ 
                C \longrightarrow aA \; | \; bB 
            \end{cases} \] 
\end{example}


% ==================================================================================================================================
% Simplification de Grammaires 

\section{Simplification de Grammaires}

Tout comme les automates, on va chercher à simplifier les grammaires. 
Or ici, pour un langage donné il n'existe pas de forme minimale de grammaire qui l'engendre.

On va donc chercher à simplifier les grammaires dans le but d'obtenir des formes dites normales 
pour réduire le nombre de dérivations à faire pour un mot donné. L'idée est de ramener les arbres de dérivation 
à des arbres binaires donc ekes dérivations sont seulement de deux formes. 
On pourra donc calculer directement la profondeur de l'arbre de dérivation de n'importe quel mot du langage
engendré en fonction de son nombre de caractères. 

Pour cela, il faut définir un certain nombre de règles qui serviront à cette simplification. 
Commençons par un règle très simple : 

\begin{definition}[Règle 0]
    On peut toujours éliminer une règle de la forme $ X \longrightarrow X$. 
\end{definition}

\subsection{Règle 1 : Suppression des epsilon-productions}

On va chercher ici à supprimer toutes les $\varepsilon$ productions qui ne produisent rien dans la 
dérivation d'un mot et prennent beaucoup de temps et de place à exécuter. 

\begin{definition}[Règle 1 : Suppression des espilon-productions]
    Soit $G$ une grammaire formelle. On définit l'algorithme suivant pour 
    supprimer toutes les $\varepsilon$-productions de $ G$ en une grammaire équivalente : 
    \begin{enumerate}
        \item On cherche récursivement toutes les variables dont $\varepsilon$ dérive
        (i.e toutes les variables qui peuvent nous donner $\varepsilon$ à la fin). 
        \item On supprimer toutes les règles de la forme $X \longrightarrow \varepsilon$. 
        \item Pour toutes les variables $X$ de la forme $X \longrightarrow w$ 
        on rajoute toutes les productions $X \longrightarrow u$ avec $ u \not = u$ et $u$ est 
        obtenu à partir de $w$ en remplaçant une ou plusieurs variables indentifiées en 1. 
    \end{enumerate}
\end{definition}

\begin{example}
    Soit $G$ une grammaire d'alphabet $\Sigma = \{a,b\}$ et $V = \{S,A,B\}$ tel que :
        \[ P : 
            \begin{cases}
                S \longrightarrow AB | aS | A \\ 
                A \longrightarrow Ab | \varepsilon \\ 
                B \longrightarrow B | AS 
            \end{cases}
        \] 
    On cherche $G'$ telle sans $\varepsilon$-productions telle que :
        \[ L(G') \cup \{\varepsilon\} = L(G) \] 
    D'après la règle 1, toutes les variables produisent une $\varepsilon$-production. 
    On obtient donc la grammaire équivalente à $\varepsilon$-production près :
        \[ P' : 
            \begin{cases}
                S \longrightarrow AB | A | B | aS | a \\ 
                A \longrightarrow Ab | b \\ 
                B \longrightarrow AS | A | S 
            \end{cases} \] 
\end{example}

\subsection{Règle 2 : Élimination des cycles}

Dans les arbres de dérivation, les cycles peuveut conduire à des dérivations infinies. 
On va donc chercher à les supprimer. 

\newpage 

\begin{definition}[Règle 2 : Élimination des cycles]
    Soit $G$ une grammaire formelle. On définit l'algorithme suivant pour 
    supprimer tous les cycles de $ G$ en une grammaire équivalente. 
    Soit un cycle de la forme : 
        \[ X_1 \longrightarrow X_{n-1} \longrightarrow X_1 \] 
    Alors on remplace dans $P$ toutes les variables $X_i \forall i \in \llbracket 1, n-1 \rrbracket$ 
    par $X_1$. 
\end{definition}

\begin{example}
    En reprennant l'exemple précédent : 
    \[ P' : 
            \begin{cases}
                S \longrightarrow AB | A | B | aS | a \\ 
                A \longrightarrow Ab | b \\ 
                B \longrightarrow AS | A | S 
            \end{cases} \] 
    On détecte un seul cycle : $ S \longrightarrow B \longrightarrow S$. 
    On applique donc l'algorithme pour obtenir : 
    \[ P'' : 
            \begin{cases}
                S \longrightarrow AS | A | aS | a \\ 
                A \longrightarrow Ab | b \\ 
            \end{cases} \] 
\end{example}

\subsection{Règle 3 : Suppression des changements de variable}

Les changement de variable dans les arbres de dérivation font perdre du temps. 
En effet, ils augmentent la profondeur de l'arbre de dérivation sans produire de lettre. 
On va donc chercher à les supprimer avec la règle 3. 

\begin{definition}[Règle 3 : Suppression des changements de variable]
    Soit $G$ une grammaire formelle. Soit une dérivation de la forme $ A \longrightarrow B \longrightarrow C$. 
    Alors on peut la remplacer en $A \longrightarrow C$. 
\end{definition}

\begin{example}
    En reprenant l'exemple précédent, on peut supprimer les changements de variable : 
    \[ P'' : 
            \begin{cases}
                S \longrightarrow AS | A | aS | a \\ 
                A \longrightarrow Ab | b \\ 
            \end{cases} \] 
    On a donc les règles de productions suivantes en remplaçant :
    \[ P''' : 
            \begin{cases}
                S \longrightarrow AS | Ab | b | aS | a \\ 
                A \longrightarrow Ab | b 
            \end{cases} \] 
\end{example}


\subsection{Forme Normale de Chomsky}

Pour l'instant, on ne peut pas encore déterminer la profondeur de l'arbre de dérivation d'un mot. 
Il faut donc définir une forme, dite Normale de Chomsky, qui va permettre cette estimation. 

\begin{definition}[Forme Normale de Chomsky]
    Soit $ G = (\Sigma, V, S, P)$ une grammaire formelle. 
    Supposons que $G$ ne contienne ni $\varepsilon$-production, si changements de variables, 
    ni cycles. On définit alors la forme normale de Chomsky de ses règles de production $P$ comme ses 
    mêmes règles de production où seules deux formes sont autorisées : 
    \begin{itemize}
        \item Les productions de lettres de la forme $ X \longrightarrow a$ 
        \item Les dédoublements de variables de la forme $ X \longrightarrow AB$ 
    \end{itemize}
\end{definition}

\begin{example}
    En reprenant l'exemplement précédent, on obtient la forme normale de Chomsky suivante : 
    \[ \overset{\sim}{P} : 
        \begin{cases}
            S \longrightarrow AS | AY | b | XS | a \\ 
            A \longrightarrow AY | b \\ 
            X \longrightarrow a \\ 
            Y \longrightarrow b 
        \end{cases} \] 
\end{example}

On en déduit donc le théorème suivant :

\begin{theorem}[Majoration de la profondeur]
    Soit $w \in L(G)$ où $G$ est une grammaire formelle sous forme normale de Chomsky. 
    Notons $p$ la profondeur de l'arbre de dérivation de $w$ dans cette grammaire $g$. 
    On a alors l'inégalité suivante : 
        \[ \boxed{p \leqslant 2 |w| -1 } \] 
\end{theorem}









\end{document}