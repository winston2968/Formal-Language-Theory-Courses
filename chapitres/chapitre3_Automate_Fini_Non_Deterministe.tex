\chapter{Automate Fini Non Déterministe}

\minitoc % Ajoute le sommaire local ici

\setlength{\parindent}{0pt}
\renewcommand{\labelitemi}{\textbullet} % Utiliser des points noirs (•)


% ==================================================================================================================================
% Introduction 

Dans le chapitre précédent nous avons vu un modèle très efficace pour vérifier l'appartenance d'un mot à un langage. 
En plus d'être facilement représentable en mémoire (i.e Python), il est facile à utiliser à la main et 
hérite de toute la théorie des graphes vue précédement ce qui en fait un très beau modèle mathématiquement parlant. 

Malgré tout cela, nous ne savons pas comment, à partir de plusieurs langages simples, constituer un automate 
reconnaissant la somme de ces langages. Nous n'avons pas défini de somme/union d'automate et celles-ci semblent 
assez difficiles vu la rigidité de notre modèle. 

Nous allons donc construire un modèle d'automate, appelé non déterministe, nous permettant de faire ces opérations 
d'union (que nous appelerons juxtaposition). Elles nous permettrons de construire des automates complexes à partir de 
somme de langages. 

% ==================================================================================================================================
% Définitions et propriétés

\section{Généralités}

\subsection{Définitions}

\begin{definition}[AFN]
    Un automate fini non déterministe $ \mathcal{A}$ est un quintuplet : 
        \[ \mathcal{A} := (Q,\Sigma, T,I,A) \] 
    tel que :
    \begin{itemize}
        \item $Q$ est l'ensemble des états de l'automate 
        \item $\Sigma$ est un alphabet 
        \item $T : Q \times \Sigma \longrightarrow \mathcal{P}(Q)$ est une application 
        \item $I \subseteq Q$ est l'ensemble des états initiaux 
        \item $A \subseteq Q$ est l'ensemble des états acceptants. 
    \end{itemize}
\end{definition}

\begin{remark}
    Plusieurs remarques concernant ce nouveau modèle. Premièrement, on remarque que l'on peut maintenant définir 
    des transitions multiples entre les états de l'automate. Deuxièmement, il existe plusieurs états initiaux. 
\end{remark}

\subsection{Lecture d'un mot}

\begin{definition}[Arbre de lecture]
    Soient $w \in \Sigma^*$, $L$ un langage sur $\Sigma$ et $ \mathcal{A}$ un automate reconnaissant $L$.
    L'arbre de lecture de $w$ par $ \mathcal{A}$ est l'arbre résultat du parcours de $ \mathcal{A}$ en fonction 
    des lettres de $w$. 
    
    Autrement dit dans l'arbre de lecture G de $w$, un noeud $a$ est le fils d'un 
    noeud $b$ si il existe une lettre $l$ de $w$ telle que $T(b,l) = a$. Une feuille de cet arbre est un état acceptant 
    ou refusant de l'automate. 
\end{definition}

\begin{definition}[Lecture acceptante]
    Soient $w \in \Sigma^*$, $L$ un langage sur $\Sigma$ et $ \mathcal{A}$ un automate reconnaissant $L$. 
    Une lecture de $w$ par $ \mathcal{A}$ est dite acceptante si \underline{il existe} un chemin 
    d'un état initial vers un état acceptant dans l'arbre de lecture de $w$ par $ \mathcal{A}$. 
\end{definition}

\begin{proposition}
    On peut dire plusieurs choses de la lecture d'un mot $w \in \Sigma^*$ par un automate $ \mathcal{A}$ :
    \begin{itemize}
        \item Si l'automate possède plusieurs états initiaux, la lecture produit un arbre de lecture pour chaque 
        état initial, nous auront donc une \underline{forêt d'arbres de lecture}. 
        \item Une lecture sera donc acceptante ssi il existe un arbre de la forêt dont au moins une des feuilles
        est un état acceptant. 
    \end{itemize}
\end{proposition}

\begin{example}
    Voyons tout cela sur un exemple. Soit $ \mathcal{A} := (Q,\Sigma, T,I,A)$ un automate fini non déterministe $ \mathcal{A}$
    sur l'alphabet $\Sigma : \{a,b\}$. Représentons notre automate sous forme de graphe orienté valué et sa table de transition :
    \begin{center}
        \begin{minipage}{0.45\textwidth} % Colonne pour le dessin
            \begin{tikzpicture}[shorten >=1pt, node distance=3cm, on grid, auto]
    
                % Définitions des styles d'états
                \node[state, initial] (1) {1};
                \node[state, accepting, right=of 1] (2) {2};
                \node[state, below=of 1] (3) {3};
                \node[state, initial, right=of 3] (4) {4};
            
                % Définition des transitions
                \path[->]
                (1) edge[above] node{$a$} (2)
                (1) edge[left] node{$a$} (3)
                (2) edge[right] node{$a$} (4)
                (3) edge[below] node{$b$} (4)
                (4) edge[sloped, below left] node{$b$} (1)
                (4) edge[loop below] node{$b$} (4);
            
            \end{tikzpicture}
        \end{minipage}%
        \hfill 
        \begin{minipage}{0.45\textwidth} % Colonne pour le tableau
            \begin{tabular}{c|c|c}
                T & a & b \\ \hline 
                $1$ & $\{2,3\}$ & X \\ \hline 
                \textcircled{2} & $4$ & X \\ \hline 
                $3$ & X & $4$ \\ \hline 
                $4$ & X & $\{1,4\}$ 
            \end{tabular}
        \end{minipage}
    \end{center}
    C'est un automate non déterministe puisqu'il contient deux transitions mutliples et deux états initiaux. 

    Posons $w := abba$, déterminons si ce mot appartient au langage $L( \mathcal{A})$. Nous allons construire 
    un seul arbre permettant d'avoir une condition suffisante de validation du mot. 

    \begin{center}
        \begin{minipage}{0.45\textwidth}
            \begin{tikzpicture}[scale=0.8, transform shape, shorten >=1pt, node distance=3cm, on grid, auto]
                
                % Racine
                \node[state, initial] (1) {1};
    
                % 1er niveau
                \node[state] (2) [below right=2cm and 2cm of 1] {2};
                \node[state] (3) [below left=2cm and 2cm of 1] {3};
    
                % 2nd niveau
                \node[state] (x1) [below=2cm of 2] {X};
                \node[state] (4) [below=2cm of 3] {4};
                
                % 3eme niveau 
                \node[state] (42) [below left=2cm and 2cm of 4] {4};
                \node[state] (12) [below right=2cm and 2cm of 4] {1};
    
                % 4 eme niveau 
                \node[state] (22) [below right=2cm and 2cm of 12] {2};
                \node[state] (32) [below left=2cm and 2cm of 12] {3};
                \node[state] (x2) [below=2cm of 42] {X}; 
    
                % Branches
                \path[->]
                (1) edge[left] node{a} (2)
                (1) edge[left] node{a} (3)
                (2) edge[left] node{b} (x1)
                (3) edge[left] node{b} (4)
                (4) edge[left] node{b} (12)
                (4) edge[left] node{b} (42)
                (12) edge[left] node{a} (22)
                (12) edge[left] node{a} (32)
                (42) edge[left] node{a} (x2);
    
            \end{tikzpicture}
        \end{minipage}
        \hfill 
        \begin{minipage}{0.45\textwidth}
            L'arbre de lecture du mot $abba$ contient un état acceptant comme feuille. 

            \vspace{1cm}

            Autrement dit, il existe un chemin menant d'un état initial à un état acceptant dans la forêt de lecture 
            de $abba$. Donc $abba \in L( \mathcal{A})$. 
        \end{minipage}        
    \end{center}
\end{example}

% ==================================================================================================================================
% Juxtaposition et construction d'un AFD

\section{Juxtaposition et construction d'un AFD}

\subsection{Juxtaposition}

Rappelons la problématique principale du chapitre. On cherche un modèle dérivant des AFD nous permettant de définir 
des opérations dessus et qui puisse être convertit algorithmiquement vers un AFD pour construire des automates 
d'un langage complexe à partir de langages plus simple. 

Autrement dit, on veut pouvoir appliquer l'opération de somme de langages sur les automates fini déterministes. 

\begin{definition}[Juxtaposition d'AFN]
    Soient $L_1$ et $L_2$ deux langages reconnus par deux automates $ \mathcal{A}_1$ et $ \mathcal{A}_2$. 
    Le langage $L_1 + L_2$ est reconnu par la \textbf{juxtaposition disjointe} de $ \mathcal{A}_1$ et $ \mathcal{A}_2$. 
\end{definition}

\begin{theorem}[Langages automatiques et stabilité]
    L'ensemble des langages automatiques est stable par somme. 
\end{theorem}

On peut maintenant, à partir de deux langages automatiques, définir le nouveau langage résultant de 
la somme des deux qui sera lui aussi automatique. Il suffit de faire la juxtaposition disjointe des deux automates
de départ. 

\subsection{Déterminisation}

\begin{theorem}[Existence et équivalence]
    Pour tout automate fini non déterministe, il existe un automate déterministe équivalent. 
\end{theorem}

Ce théorème est peut être un peu obscur mais permet de dire qu'il est toujours possible de passer d'un 
automate fini non déterministe (obtenu par exemple par juxtaposition) à un automate fini déterministe
qui reconnaisse le même langage. En tout cas, il nous dit qu'il en existe un...

\vspace{0.3cm}

L'intérêt de vouloir repasser chez les automates fini déterministes vient du fait que la lecture d'un mot par 
un AFN est de complexité exponentielle alors que la lecture d'un mot par un AFD est polynômiale... 
Lors de la vérification syntaxique de très long mots pour des langages très complexes, cela fait une différence 
cruciale pour la compilation. 

Ce processus est appelé \textbf{déterminisation} d'un AFN. 

\begin{proposition}
    Soit $ \mathcal{A} = (Q, \Sigma, T, I, A)$ un AFN. On cherche à construire un AFD $ \mathcal{A}'$ équivalent à $ \mathcal{A}$. 
    L'idée est de raisonner sur l'application $T : Q \times \Sigma \longrightarrow \Q$. Dans un AFN, cette application 
    n'est pas injective, on va donc poser une nouvelle application dans l'espace quotient de $Q \times \Sigma$ par le noyau de $T$. 
    Nous obtiendrons donc une application injective et donc un AFD. 
\end{proposition}

\newpage 

\begin{example}[Déterminisation]
    Soit $ \mathcal{A}$ l'automate défini sur l'alphabet $\Sigma = \{ a,b \}$ non déterministe et sa table 
    de transition suivants : 

    \begin{center}
        \begin{minipage}{0.45\textwidth} % Colonne pour le dessin
            \begin{tikzpicture}[shorten >=1pt, node distance=3cm, on grid, auto]
    
                % Définitions des styles d'états
                \node[state, initial] (1) {1};
                \node[state, accepting, right=of 1] (2) {2};
                \node[state, initial, below=of 1] (3) {3};
                \node[state, right=of 3] (4) {4};
            
                % Définition des transitions
                \path[->]
                (1) edge[above] node{$a$} (2)
                (1) edge[left] node{$a$} (3)
                (2) edge[right] node{$a$} (4)
                (3) edge[below] node{$b$} (4)
                (4) edge[sloped, below left] node{$b$} (1)
                (4) edge[loop below] node{$b$} (4)
                (2) edge[loop above] node{$b$} (2)
                (3) edge[loop below] node{$a$} (3);
            
            \end{tikzpicture}
        \end{minipage}%
        \hfill 
        \begin{minipage}{0.45\textwidth} % Colonne pour le tableau
            \begin{tabular}{c|c|c}
                T & a & b \\ \hline 
                $1$ & $\{2,3\}$ & X \\ \hline 
                \textcircled{2} & $4$ & $2$ \\ \hline 
                $3$ & $3$ & $4$ \\ \hline 
                $4$ & X & $\{1,4\}$ 
            \end{tabular}
        \end{minipage}
    \end{center}

    Déterminisons cet automate. Pour cela, nous allons renommer tous les états de l'automate en prenant en 
    compte les ensembles. 
    L'algorithme consiste donc à construire la table de transition du nouvel automate. 
    Pour chaque itération (i.e ajout d'une ligne dans la table), on effectue un parcours en largeur du nouvel automate 
    pour "découvrir" de nouveau état. On créé ainsi un "automate des parties". 

    \begin{figure}[h]
    \begin{center}
        \begin{minipage}{0.45\textwidth}
            \begin{tabular}{c|c|c}
                & a & b \\ \hline 
                $I = \{1,3\}$ & $II$ & $III$ \\ 
                $II = \{2,3\}$ & $IV$ & $V$ \\ 
                $III = \{4\}$ & - & $VI$ \\ 
                $IV = \{3,4\}$ & $VII$ & $VI$ \\ 
                $V = \{2,4\}$ & $III$ & $ VIII$ \\ 
                $VI = \{1,4\}$ & $ II$ & $VI$ \\ 
                $VII = \{3\}$ & $VII$ & $III$ \\ 
                $VIII = \{1,2,4\}$ & $IX$ & $VIII$ \\ 
                $IX = \{2,3,4\}$ & $IV$ & $VIII$ 
            \end{tabular}
            \caption{Table de l'AFD}
        \end{minipage}
        \hfill 
        \begin{minipage}{0.45\textwidth}
            \begin{tikzpicture}[scale=0.8, transform shape, shorten >=1pt, node distance=3cm, on grid, auto]
                % 1er niveau
                \node[state] (I) {I};

                % 2nd niveau
                \node[state] (II) [below left=2cm and 2cm of I] {II};
                \node[state] (III) [below right=2cm and 2cm of I] {III};

                % 3eme niveau
                \node[state] (IV) [below left=2cm and 2cm of II] {IV};
                \node[state] (V) [below right=2cm and 2cm of II] {V};
                \node[state] (VI) [below=2cm of III] {VI};

                % 4eme niveau
                \node[state] (VII) [below=2cm of IV] {VII};
                \node[state] (VIII) [below=2cm of V] {VIII}; 

                % 5eme niveau
                \node[state] (IX) [below=2cm of VIII] {IX};
                
                % Branches 
                \path[-]
                (I) edge[left] node{a} (II)
                (I) edge[left] node{b} (III)
                (II) edge[left] node{a} (IV)
                (II) edge[left] node{b} (V)
                (III) edge[left] node{b} (VI)
                (IV) edge[left] node{} (VII)
                (V) edge[left] node{} (VIII)
                (VIII) edge[left] node{} (IX);
            \end{tikzpicture}
            \caption{Automate des parties}
        \end{minipage}
    \end{center} 
    \end{figure}
\end{example}







