\chapter{Automate Fini Déterministe}

\minitoc % Ajoute le sommaire local ici

\setlength{\parindent}{0pt}
\renewcommand{\labelitemi}{\textbullet} % Utiliser des points noirs (•)


% ==================================================================================================================================
% Introduction 

Comme abordé dans le chapitre précédent, on cherche une méthode pratique et efficace pour déterminer si un mot appartient 
à un langage ou pas. On veut donc un modèle qui soit d'une part très pratique mathématiquement pour nous permettre de 
démontrer des choses dessus mais aussi facilemet implémentable algorithmiquement. 

Alerte Spoiler : de solides connaissances en théorie des graphes seront plus qu'utiles...


% ==================================================================================================================================
% Définition et représentation

\section{Définition et représentation}

\begin{definition}[Automate fini déterministe]
    Un Automate Fini Déterministe est un quintuplet :
        \[ \boxed{ \mathcal{A} = (\Sigma, Q, T, q_0, A) } \] 
    où :
    \begin{itemize}
        \item $\Sigma$ est un alphabet 
        \item $Q$ est un ensemble fini d'états (souvent une partie finie de $\N$)
        \item $T : Q \times \Sigma \longrightarrow Q$ est une application qui, à un état et une lettre associe un autre état. 
        \item $q_0$ un état initial
        \item $A \subseteq Q$ les états acceptants
    \end{itemize}
\end{definition}

On représentera ainsi un automate fini déterministe de plusieurs façons en fonction de son utilisation :
\begin{itemize}
    \item \textbf{Mathématique : } $\mathcal{A} = (\Sigma, Q, T, q_0, A) $
    \item \textbf{Table de Transition : } Elle va permettre de trouver rapidement les différents types d'états. 
    \item \textbf{Sagitale : } Sous forme de graphe 
    \item \textbf{En Python : } Nous représenterons les automates finis déterministes sous la forme de quintuplet aussi. 
\end{itemize}
Regardons en détail ces différentes représentations au travers d'un exemple. 

\newpage 


\begin{example}
    Soit $\mathcal{A} = (\Sigma, Q, T, q_0, A)$ un automate fini déterministe. 

    \textbf{Mathématiquement} nous avons :
    \begin{itemize}
        \item $Q = \{1,2,3\}$ 
        \item $\Sigma = \{a,b\}$ 
        \item $q_0 = 1$ 
        \item $A = \{3\}$ 
    \end{itemize}
    
    La représentation \textbf{sagitale}  de notre automate sera :
        
        \begin{center}
            \begin{figure}[h]
                \centering
                \begin{tikzpicture}[shorten >=1pt, node distance=2cm, on grid, auto]
    
                    % Définitions des styles d'états
                    \node[state, initial] (1) {1};
                    \node[state, right=of 1] (2) {2};
                    \node[state, accepting, right=of 2] (3) {3};
                
                    % Définition des transitions
                    \path[->]
                    (1) edge[above] node{a} (2)
                    (2) edge[above] node{b} (3)
                    (2) edge[loop above] node{a} ()
                    (1) edge[loop above] node{b} ()
                    (3) edge[loop above] node{a} ()
                    (3) edge[loop below] node{b} ();
                
                \end{tikzpicture}
            \end{figure}
        \end{center}

    On représente de façon doublement cerclée les états acceptants. Les états sont les sommets du graphe. 
    Les arcs valués sont les antécédants/images de la fonction $T$. 

    Enfin, la \textbf{table de transition} de l'automate est représentée par le tableau suivant :
    \begin{multicols}{2}
        \begin{center}
            \begin{tabular}{c|c|c}
                Q/$\Sigma$ & a & b \\ \hline 
                1 & 2 & 3 \\ \hline 
                2 & 2 & 3 \\ \hline 
                \textbf{3} & 3 & 3 
            \end{tabular}
        \end{center}

        La première ligne présente les lettres de l'alphabet et la première colonne les différents états. 
        Pour chaque état, le tableau donne l'état obtenu en fonction de la lettre suivante lue. 
    \end{multicols}
\end{example}

% ==================================================================================================================================
% Mots et Langage Automatique

\section{Mot et Langage Automatique}

\subsection{Lecture d'un mot}

\begin{definition}[Lecture d'un mot]
    Soit $\Sigma$ un dictionnaire, $l \in \Sigma$ et $\mathcal{A}$ un automate. 
    On lit la lettre $l$ en dérivant d'un état $q \in Q$ vers un état $q' \in Q$ et si $T(q,l) = q'$. 
    On notera la lecture d'un mot de longueur $p$ par la lecture successive de ses lettres :
        \[ q_1 \;  \overset{l_1}{\longrightarrow} \; q_2 \; \dots \; q_{p-1} \; \overset{l_p}{\longrightarrow} \; q_p \] 
\end{definition}

Concrètement, pour la lecture d'un mot, on va partir de l'état initial et en fonction des valeurs de l'état courant et de 
la lettre lue, on va "bouger" d'un état (sommet) à un autre.

\begin{definition}[Mot refusé]
    Un mot $w \in \Sigma^*$ est dit refusé par un automate $\mathcal{A}$ si sa lecture à partir de l'état initial 
    se termine sur un état refusant ou ne se termine pas. Dans le cas contraire, $w$ est dit accepté. 
\end{definition}

\begin{definition}[Langage d'un Automate]
    Le langage d'un automate $\mathcal{A}$ est l'ensemble des mots acceptés par l'automate. 
    On le note $L(\mathcal{A})$. 
\end{definition}

On parle de \textbf{langage automatique} si il est reconnaissable par un automate. 
Deux automates sont dits \textbf{équivalents} si ils reconnaissent le même langage. 

\subsection{Automate Complet}

\begin{definition}[Automate Complet]
    Un automate $\mathcal{A}$ est dit complet si 
        \[ \forall i \in Q, \forall l \in \Sigma, \quad T(i,l) \text{ est défini} \] 
    Autrement dit, un automate est dit complet si pour toute lettre et pour tout état fixés, il est possible de 
    changer d'état dans l'automate.  
\end{definition} 

\begin{definition}[Puit]
    Un état $q \in Q$ est un puit ssi 
        \[ \forall l \in \Sigma, \quad T(q,l) = q \]
    Un puit est un état duquel on ne peut sortir. 
\end{definition}

On définit aussi la notion de piège comme un puit refusant (i.e un puit dont l'état est refusant). 

\subsection{Complémentaire}

Puisque la notion de complémentaire existe pour les langages et que les automates semblent très étroitement liés 
aux langages, on peut se demander si un automate peut admettre un complémentaire... 

Soit $\mathcal{A}$ un automate associé à un langage L. On cherche $\mathcal{A}' = \overline{\mathcal{A}}$. 
    \[ w \in \overline{\mathcal{A}} \iff w \not \in L \iff w \not \in L(\mathcal{A}) \] 
Il semble falloir que $\mathcal{A}$ soit complet. Si c'est le cas, on pourrait inverser $\mathcal{A}$ en inversant 
les sommets accpetants/refusants. 

\begin{proposition}
    Tout automate fini peut être complété par des puits refusants en un automate complet. 
\end{proposition}

\begin{theorem}[Complémentarité]
    L'ensemble des automates est stable par complémentarité. 
\end{theorem}


% TODO : Ajouter section émonder un automate 