\chapter{Opérations entre automates}

\minitoc % Ajoute le sommaire local ici

\setlength{\parindent}{0pt}
\renewcommand{\labelitemi}{\textbullet} % Utiliser des points noirs (•)


% ==================================================================================================================================
% Introduction 

Les chapitres précédents nous ont montré que les $\varepsilon$-transitions et les transitions multiples 
nous permettent de modéliser plus facilement des langages complexes. 
Nous allons donc définir des opérations sur les automates, analogues à celles sur les langages. 
Elles nous permettront, à partir de la construction d'un langage par opérations, de construire son automate par 
opérations aussi. 

\vspace{0.3cm}

\textbf{Problématique : } Soient $L_1$ et $L_2$ deux langages reconnus respectivement par $ \mathcal{A}_2$ et $ \mathcal{A}_2$. 
Soit $T$ une opération entre langages. Comment construire un automate reconnaissant $L_1 T L_2$ ? 

% ==================================================================================================================================
% Langage Elémentaires 

\section{Langages Elémentaires}

Pour définir des opérations sur les automates, et ainsi, construire un automate par opérations pour un langage 
lui-même construit par opérations, nous allons avoir besoin de définir des langages élémentaires. 
Ces langages seront reconnus par des automates fixés, que nous connaissons d'avance. Nous en choisissons un nombre fini 
pour pouvoir les stocker en mémoire. Ils vont représenter les briques de base nous permettant de construire des automates plus 
complexes par la suite. 

\begin{proposition}[Langages Elémentaires]
    Soit $\Sigma = \{a,b\}$ un alphabet. On définit les langages élémentaires de $\Sigma$ comme :
    \begin{itemize}
        \item $L = \emptyset$ reconnu par l'automate : 
            \begin{center}
                \begin{tikzpicture}[shorten >=1pt, node distance=3cm, on grid, auto]
                    \node[state, initial] (1) {$1$};
                \end{tikzpicture}
            \end{center}
        \item $L = \{\varepsilon\}$ reconnu par l'automate :
            \begin{center}
                \begin{tikzpicture}[shorten >=1pt, node distance=3cm, on grid, auto]
                    \node[state, initial, accepting] (1) {$1$};
                \end{tikzpicture}
            \end{center}
        \item $L = \{a\}$ reconnu par le langage :
            \begin{center}
                \begin{tikzpicture}[shorten >=1pt, node distance=3cm, on grid, auto]
                    \node[state, initial] (1) {$1$};
                    \node[state, accepting, right of =1] (2) {$2$}; 
                    \path[->]
                    (1) edge[above] node{$a$} (2);
                \end{tikzpicture}
            \end{center}
        \item $L = \{a_1 \dots a_n\}$ reconnu par le langage :
            \begin{center}
                \begin{tikzpicture}[shorten >=1pt, node distance=3cm, on grid, auto]
                    \node[state, initial] (1) {$1$};
                    \node[state, right of =1] (2) {$2$}; 
                    \node[state, right of=2] (3) {$a_n$};
                    \node[state, accepting, right of=3] (4) {$n$};
                    \path[->]
                    (1) edge[above] node{$a_1$} (2)
                    (2) edge[above] node{$a_2$} (3)
                    (3) edge[above] node{$\dots$} (4);
                \end{tikzpicture}
            \end{center}
    \end{itemize}
\end{proposition}

% ==================================================================================================================================
% Langage complémentaire

\section{Automate Complémentaire}

A partir d'un langage, on peut définir son langage complémentaire. De même, on peut définir l'automate complémantaire 
reconnaissant ce langage. 

\begin{definition}[Automate Complémentaire]
    Soit $L$ un langage reconnu un automate $ \mathcal{A} = (Q,\Sigma,T,q_0,A)$ \textbf{complet}. 
    L'automate reconnaissant le langage complémentaire de $L$ est $ \overline{ \mathcal{A}} = (Q,\Sigma,T,q_0,Q \backslash A)$. 
\end{definition}

\begin{remark}
    Attention, pouvoir "passer au complémentaire" pour un automate, il faut que celui-ci soit complet. 
\end{remark}

% ==================================================================================================================================
% Somme d'automates

\section{Somme d'automates}

\begin{definition}[Somme d'automates]
    Soient $L_1$ et $L_2$ deux langages respectivement reconnus par $ \mathcal{A}_1$ et $ \mathcal{A}_2$. 
    L'automate reconnaissant $L_1 + L_2$ est l'automate fini non déterministe construit par \textbf{l'union disjointe}
    de $ \mathcal{A}_1$ et $ \mathcal{ A}_2$. On l'appelle \textbf{automate somme} des automates $ \mathcal{A}_1$ 
    et $ \mathcal{A}_2$. 
\end{definition}

Cet automate n'est pas déterministe puisqu'il contient deux états initiaux mais que l'on peut déterminiser. 

\begin{example}
    Soient les langages suivants sur $\Sigma = \{a,b\}$ :
    \begin{align*}
        L_1 &= \{ \text{mots de } \Sigma \text{ terminant par } a \} \\ 
        L_2 &= \{ \text{mots de } \Sigma \text{contenant un nombre pair de } a \}
    \end{align*}

    Ces langages sont reconnus par les automates suivants :

    \begin{center}
        \begin{minipage}{0.45\textwidth}
            \begin{tikzpicture}[shorten >=1pt, node distance=3cm, on grid, auto]
                \node[state, initial] (1) {$1$};
                \node[state, right of=1, accepting] (2) {$2$};
                
                \path[->]
                (1) edge[above, bend right] node{$a$} (2)
                (2) edge[above, bend right] node{$b$} (1)
                (2) edge[loop right] node{$a$} (2)
                (1) edge[loop above] node{$b$} (1); 
            \end{tikzpicture}
        \end{minipage}%
        \hfill 
        \begin{minipage}{0.45\textwidth}
            \begin{tikzpicture}[shorten >=1pt, node distance=3cm, on grid, auto]
                \node[state, initial] (1) {$1$};
                \node[state, right of=1, accepting] (2) {$2$};
                
                \path[->]
                (1) edge[above, bend right] node{$a$} (2)
                (2) edge[above, bend right] node{$a$} (1)
                (2) edge[loop right] node{$b$} (2)
                (1) edge[loop above] node{$b$} (1); 
            \end{tikzpicture}
        \end{minipage}
    \end{center}

    On construit l'automate $ \mathcal{A}$ comme la juxtaposition disjointe des deux automates :
    \begin{center}
        \begin{minipage}{0.45\textwidth}
            \begin{tikzpicture}[shorten >=1pt, node distance=3cm, on grid, auto]
                \node[state, initial] (1) {$1$};
                \node[state, right of=1, accepting] (2) {$2$};
                
                \path[->]
                (1) edge[above, bend right] node{$a$} (2)
                (2) edge[above, bend right] node{$b$} (1)
                (2) edge[loop right] node{$a$} (2)
                (1) edge[loop above] node{$b$} (1); 
            \end{tikzpicture}
        \end{minipage}%
        \hfill 
        \begin{minipage}{0.45\textwidth}
            \begin{tikzpicture}[shorten >=1pt, node distance=3cm, on grid, auto]
                \node[state, initial] (3) {$3$};
                \node[state, right of=3, accepting] (4) {$4$};
                
                \path[->]
                (3) edge[above, bend right] node{$a$} (4)
                (4) edge[above, bend right] node{$a$} (1)
                (4) edge[loop right] node{$b$} (4)
                (3) edge[loop above] node{$b$} (3); 
            \end{tikzpicture}
        \end{minipage}
    \end{center}

    Que l'on doit ensuite déterminiser... 
\end{example}

Les définitions de langages élémentaires nous permettent de savoir que tout langage réduit à un mot est automatique. 
On en déduit le théorème suivant : 

\begin{theorem}[Langage Fini]
    Tout langage fini est automatique. 
\end{theorem}


% ==================================================================================================================================
% Intersection d'automates

\section{Intersection d'Automates}

\begin{proposition}
    Soient $L_1$ et $L_2$ reconnus par les automates fini déterministes suivants $ \mathcal{A}_1$ et $ \mathcal{A}_2$. 
    On a alors : 
        \[ \overline{L_1 \cap L_2} = \overline{L_1} + \overline{L_2} \]
    On peut donc en conclure que : 
        \[ \boxed{L_1 \cap L_2 = \overline{\overline{L_1} + \overline{L_2}}} \]  
    Ainsi, l'intersection de deux automates peut être construite par somme et complémentarité. 
\end{proposition}

En pratique nous utuliseront plutôt l'automate des couples : 

\begin{definition}[Automate des Couples]
    Soient $ \mathcal{A}_1 = \{Q_1, \Sigma, T_1, Q_0, A_1\}$ reconnaissant le langage $L_1$ et 
    $ \mathcal{A}_2 = \{Q_2, \Sigma, T_2, q_0', A_2\}$ reconnaissant le langage $L_2$. On définit 
    l'automate des couples : 
        \[ A = \{Q_1 \times Q_2, \Sigma, T, (q_0,q_0'), A_1 \times A_2\} \] 
    reconnaissant le langage $L_1 \cap L_2$ où 
        \[ \forall i \in \Sigma, \forall q_1, q_2 \in Q_1, \forall q_1', q_2' \in Q_2 \text{ tels que } q_2 = T_1(q_1,j) \text{ et } q_2' = T(q_1',i) \]
        \[ \text{alors } T'((q_1, q_1'), i) = (q_2, q_2') \] 
\end{definition}

\begin{proposition}
    Si $ \mathcal{A}_1$ possède $n \in \N$ états et que $ \mathcal{A}_2$ possède $p \in \N$ états, alors 
    $ \mathcal{A}_1 \cap \mathcal{A}_2$ possède $n \times p$ états. 
    Pour de gros automates, cette méthode peut donc engendrer des très gros. 
    Même si la façon de les construire est assez simple et ressemble beaucoup à la déterminisation. 

    L'automate obtenu est, de plus, déterministe. 
\end{proposition}


% ==================================================================================================================================
% Différences d'automates

\section{Différence d'Automates}

\begin{proposition}
    Soient $ \mathcal{A}_1$ et  $ \mathcal{A}_2$ deux automates reconnaissant respectivement les langages $L_1$ et $L_2$. 
    Pour reconnaître le langages $L_1 \backslash L_2$, on peut simplement construire l'automate reconnaissant :
        \[ L_1 \backslash L_2 = L_1 \cap \overline{L_2} \] 
\end{proposition}

% ==================================================================================================================================
% Langages Automatiques 

\newpage

\section{Langages Automatiques}

Essayons maintenant de déduire des conditions nécessaire pour qu'un langage soit automatique. 
D'après ce que l'on a vu grâce aux opérations, on peut déjà énoncer la proposition suivante : 
\begin{proposition}
    Les langages réguliers sont tous automatiques. 
\end{proposition}

Proposition que nous élargirons plus tard grâce au théorème de Kleene. 

\subsection{Théorème de pompage}

\begin{theorem}[Pompage (faible)]
    Soit $L$ un langage. Supposons $L$ automatique. 
    Soit $N \in \N$, alors pour tout $w \in L$ tel que $ |w| \leqslant N$, $w$ admet une décomposition 
    de la forme : 
        \[\exists w_1, w_2, w_3 \in L, w = w_1 w_2 w_3 \quad \text{avec } w_2 \not  = \varepsilon \]
    telle que cette décomposition soit gonflable :
        \[ \text{ i.e } \forall k \in \N, w_1 w_2 ^k w_3 \in L \]
\end{theorem} 

Ce théorème n'est pas idéal pour montrer qu'un langage est automatique. En revanche, sa contraposé de la forme :
    \[ \text{Non Pompable } \Longrightarrow \text{Non Automatique} \]  
Est en pratique très utilisée pour montrer qu'un langage n'est pas automatique. 

\begin{proposition}
    Ainsi, soit $L$ un langage. Pour montrer que $L$ n'est pas pompable, on montrera que 
    pour tout $N \in \N$, il existe $w \in L$ tel que $ |w| \geqslant N$ qui admette une décomposition de la forme : 
        \[ w = w_1 w_2 w_3 \quad \quad \text{avec } w_2 \not  = \varepsilon \] 
    telle que : 
        \[ \exists k \in \N, w_1 w_2^k w_3 \not \in L \] 
\end{proposition}

